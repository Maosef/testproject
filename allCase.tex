%UNTIL 185 
\documentclass[twoside,11pt]{article} 
\usepackage[pdftex]{hyperref} 
\hypersetup{ 
 	COLORLINKS = TRUE, 
	ALLCOLORS = {RED}, 
} 
\usepackage{color} 
\usepackage{amsmath} 
\usepackage{amssymb} 
\usepackage{xparse} 
\usepackage{cite} 
\renewcommand\citeform[1]{K#1} 
\setlength{\textwidth}{16cm} 
\setlength{\textheight}{21cm} 
\setlength{\topmargin}{0cm} 
\setlength{\oddsidemargin}{0.2mm} 
\setlength{\evensidemargin}{0.2mm} 
 
\newcommand\sa{\smallskipamount} 
\newcommand\sLP{\\[\sa]} 
\newcommand\sPP{\\[\sa]\indent} 
\newcommand\ba{\bigskipamount} 
\newcommand\bLP{\\[\ba]} 
\newcommand\CC{\mathbb{C}} 
\newcommand\RR{\mathbb{R}} 
\newcommand\ZZ{\mathbb{Z}} 
\newcommand\al\alpha 
\newcommand\be\beta 
\newcommand\ga\gamma 
\newcommand\de\delta 
\newcommand\tha\theta 
\newcommand\la\lambda 
\newcommand\om\omega 
\newcommand\Ga{\Gamma} 
\newcommand\half{\frac12} 
\newcommand\thalf{\tfrac12} 
\newcommand\iy\infty 
\newcommand\wt{\widetilde} 
\newcommand\const{{\rm const.}\,} 
\newcommand\Zpos{\ZZ_{>0}} 
\newcommand\Znonneg{\ZZ_{\ge0}} 
\newcommand{\hyp}[5]{\,\mbox{}_{#1}F_{#2}\!\left( 
  \genfrac{}{}{0pt}{}{#3}{#4};#5\right)} 
\newcommand{\qhyp}[5]{\,\mbox{}_{#1}\phi_{#2}\!\left( 
  \genfrac{}{}{0pt}{}{#3}{#4};#5\right)} 
\newcommand\LHS{left-hand SIDE} 
\newcommand\RHS{right-hand SIDE} 
\renewcommand\Re{{\rm Re}\,} 
\renewcommand\Im{{\rm Im}\,} 
 
\NewDocumentCommand\mycite{m G}{% 
  \IfNoValueTF{#2} 
    {[\hyperlink{#1}{#1}]} 
    {[\hyperlink{#1}{#1}, #2]}% 
} 
\newcommand\mybibitem[1]{\bibitem[#1]{#1}\hypertarget{#1}{}} 
 
%\NewDocumentCommand{\myciteKLS}{m G}{% 
%  \IfNoValueTF{#2} 
%    {\hyperlink{KLS#1}{[#1]}} 
%    {\hyperlink{KLS#1}{[#1, #2]}}% 
%} 
\NewDocumentCommand{\myciteKLS}{m G}{% 
  \IfNoValueTF{#2} 
    {[\hyperlink{KLS#1}{#1}]} 
    {[\hyperlink{KLS#1}{#1}, #2]}% 
} 
\newcommand\mybibitemKLS[1]{\bibitem[#1]{#1}\hypertarget{KLS#1}{}} 
 
\begin{document} 
 
\title{Additions TO THE FORMULA LISTS IN 
``hYPERGEOMETRIC ORTHOGONAL POLYNOMIALS AND THEIR $Q$-ANALOGUES'' 
BY kOEKOEK, lESKY AND sWARTTOUW} 
\author{Tom h. kOORNWINDER} 
\date{June 19, 2015} 
\maketitle 
\begin{abstract} 
tHIS REPORT GIVES A RATHER ARBITRARY CHOICE OF FORMULAS FOR 
($Q$-)HYPERGEOMETRIC ORTHOGONAL POLYNOMIALS WHICH THE AUTHOR MISSED 
WHILE CONSULTING cHAPTERS 9 AND 14 IN THE BOOK 
``hYPERGEOMETRIC ORTHOGONAL POLYNOMIALS AND THEIR $Q$-ANALOGUES'' 
BY kOEKOEK, lESKY AND sWARTTOUW. tHE SYSTEMATICS OF THESE CHAPTERS WILL BE FOLLOWED 
HERE, IN PARTICULAR FOR THE NUMBERING OF SUBSECTIONS AND OF REFERENCES. 
\end{abstract} 
% 
\subsection*{Introduction} 
\label{sec_intro} 
tHIS REPORT CONTAINS SOME FORMULAS ABOUT ($Q$-)HYPERGEOMETRIC 
ORTHOGONAL POLYNOMIALS WHICH i MISSED BUT WANTED TO USE 
WHILE CONSULTING cHAPTERS 9 AND 14 IN THE BOOK \mycite{KLS}: 
\sLP 
r. kOEKOEK, p.Aa. lESKY AND r.Af. sWARTTOUW, 
{\em hYPERGEOMETRIC ORTHOGONAL POLYNOMIALS AND THEIR $Q$-ANALOGUES}, 
sPRINGER-vERLAG, 2010. 
\sLP 
tHESE CHAPTERS FORM TOGETHER THE (SLIGHTLY EXTENDED) SUCCESSOR OF THE REPORT 
\sLP 
r.AkOEKOEK AND  r.Af. sWARTTOUW, 
{\em tHE aSKEY-SCHEME OF HYPERGEOMETRIC ORTHOGONAL 
POLYNOMIALS AND ITS $Q$-ANALOGUE}, 
rEPORT 98-17, fACULTY OF tECHNICAL mATHEMATICS AND iNFORMATICS, 
dELFT uNIVERSITY OF tECHNOLOGY, 1998; 
\url{http://aw.twi.tudelft.nl/~koekoek/askey/}. 
\sPP 
cERTAINLY THESE CHAPTERS GIVE COMPLETE LISTS OF FORMULAS OF SPECIAL TYPE, FOR INSTANCE 
ORTHOGONALITY RELATIONS AND THREE-TERM RECURRENCE RELATIONS. bUT OUTSIDE THESE NARROW 
CATEGORIES THERE ARE MANY OTHER 
FORMULAS FOR ($Q$-)ORTHOGONAL POLYNOMIALS WHICH ONE WANTS TO HAVE AVAILABLE. 
oFTEN ONE CAN FIND THE DESIRED FORMULA IN ONE OF THE 
\hyperref[sec_ref1]{standard REFERENCES} LISTED AT THE END OF THIS REPORT. 
sOMETIMES IT IS ONLY AVAILABLE IN A JOURNAL OR A LESS COMMON MONOGRAPH. 
jUST FOR MY OWN COMFORT, i HAVE BROUGHT TOGETHER SOME OF THESE FORMULAS. 
tHIS WILL POSSIBLY ALSO BE HELPFUL FOR SOME OTHER USERS. 
 
uSUALLY, ANY TYPE OF FORMULA i GIVE FOR A SPECIAL CLASS OF POLYNOMIALS, WILL SUGGEST 
A SIMILAR FORMULA FOR MANY OTHER CLASSES, BUT i HAVE NOT AIMED AT COMPLETENESS 
BY FILLING IN A FORMULA OF SUCH TYPE AT ALL PLACES. tHE RESULTING CHOICE OF FORMULAS IS 
RATHER ARBITRARY, JUST DEPENDING ON THE FORMULAS WHICH i HAPPENED TO NEED OR WHICH RAISED MY INTEREST. 
fOR EACH FORMULA i GIVE  A SUITABLE REFERENCE OR i SKETCH A 
PROOF. 
iT IS MY INTENTION TO GRADUALLY EXTEND THIS COLLECTION OF FORMULAS. 
% 
\subsection*{Conventions} 
\label{sec_conv} 
tHE (X.Y) AND (X.Y.Z) TYPE SUBSECTION NUMBERS, THE 
(X.Y.Z) TYPE FORMULA NUMBERS, AND THE [X] TYPE CITATION NUMBERS 
REFER TO \mycite{KLS}. 
tHE (X) TYPE FORMULA NUMBERS REFER TO THIS MANUSCRIPT AND THE [kX] TYPE CITATION NUMBERS REFER TO CITATIONS WHICH ARE NOT IN \mycite{KLS}. 
sOME STANDARD REFERENCES LIKE \mycite{DLMF} 
ARE GIVEN BY SPECIAL ACRONYMS. 
 
$n$ IS ALWAYS A POSITIVE INTEGER. aLWAYS ASSUME $N$ TO BE A NONNEGATIVE 
INTEGER OR, IF $n$ IS PRESENT, TO BE IN $\{0,1,\ldots,N\}$. 
tHROUGHOUT ASSUME $0<Q<1$. 
 
fOR EACH FAMILY THE COEFFICIENT OF THE TERM OF HIGHEST DEGREE OF THE 
ORTHOGONAL POLYNOMIAL OF DEGREE $N$ CAN BE FOUND IN \mycite{KLS} AS THE 
COEFFICIENT OF $P_N(X)$ IN THE FORMULA AFTER THE MAIN FORMULA UNDER 
THE HEADING ``nORMALIZED rECURRENCE rELATION". iF THAT MAIN FORMULA IS NUMBERED 
AS (X.Y.Z) THEN i WILL REFER TO THE SECOND FORMULA AS (X.Y.ZB). 
 
iN THE NOTATION OF $Q$-HYPERGEOMETRIC ORTHOGONAL POLYNOMIALS WE 
WILL FOLLOW THE CONVENTION THAT THE PARAMETER LIST AND $Q$ ARE SEPARATED 
BY `$\,|\,$' IN THE CASE OF A $Q$-QUADRATIC LATTICE (FOR INSTANCE 
\hyperref[sec14.1]{Askey-Wilson}) 
AND BY `;' IN THE CASE OF A $Q$-LINEAR LATTICE (FOR INSTANCE 
\hyperref[sec14.5]{big $Q$-jACOBI}). tHIS CONVENTION IS MOSTLY FOLLOWED 
IN \mycite{KLS}, BUT NOT EVERYWHERE, SEE FOR INSTANCE 
\hyperref[sec14.20]{little $Q$-lAGUERRE / wALL}. 
% 
\subsection*{Acknowledgement} 
mANY THANKS TO hOWARD cOHL FOR HAVING CALLED MY ATTENTION SO OFTEN TO TYPOS AND 
INCONSISTENCIES. 
% 
\newpage 
\subsection*{Contents} 
\hyperref[sec_intro]{Introduction}\\ 
\hyperref[sec_conv]{Conventions}\\ 
\hyperref[sec_general]{Generalities} 
\sLP 
\hyperref[sec9.1]{9.1 Wilson}\\ 
\hyperref[sec9.2]{9.2 Racah}\\ 
\hyperref[sec9.3]{9.3 cONTINUOUS DUAL Hahn}\\ 
\hyperref[sec9.4]{9.4 cONTINUOUS Hahn}\\ 
\hyperref[sec9.5]{9.5 Hahn}\\ 
\hyperref[sec9.6]{9.6 dUAL Hahn}\\ 
\hyperref[sec9.7]{9.7 Meixner-Pollaczek}\\ 
\hyperref[sec9.8]{9.8 jACOBI} 
 
\hyperref[sec9.8.1]{9.8.1 gEGENBAUER / uLTRASPHERICAL} 
 
\hyperref[sec9.8.2]{9.8.2 Chebyshev}\\ 
\hyperref[sec9.9]{9.9 pSEUDO jACOBI (OR Routh-Romanovski)}\\ 
\hyperref[sec9.10]{9.10 Meixner}\\ 
\hyperref[sec9.11]{9.11 Krawtchouk}\\ 
\hyperref[sec9.12]{9.12 Laguerre}\\ 
\hyperref[sec9.14]{9.14 Charlier}\\ 
\hyperref[sec9.15]{9.15 hERMITE} 
\sLP 
\hyperref[sec14.1]{14.1 Askey-Wilson}\\ 
\hyperref[sec14.2]{14.2 $q$-Racah}\\ 
\hyperref[sec14.3]{14.3 cONTINUOUS DUAL $q$-Hahn}\\ 
\hyperref[sec14.4]{14.4 cONTINUOUS $q$-Hahn}\\ 
\hyperref[sec14.5]{14.5 bIG $q$-Jacobi}\\ 
\hyperref[sec14.7]{14.7 dUAL $q$-Hahn}\\ 
\hyperref[sec14.8]{14.8 Al-Salam-Chihara}\\ 
\hyperref[sec14.9]{14.9 $q$-Meixner-Pollaczek}\\ 
\hyperref[sec14.10]{14.10 cONTINUOUS $Q$-jACOBI} 
 
\hyperref[sec14.10.1]{14.10.1 cONTINUOUS $Q$-ULTRASPHERICAL / Rogers}\\ 
\hyperref[sec14.11]{14.11 bIG $q$-Laguerre}\\ 
\hyperref[sec14.12]{14.12 lITTLE $q$-Jacobi}\\ 
\hyperref[sec14.14]{14.14 qUANTUM $q$-Krawtchouk}\\ 
\hyperref[sec14.16]{14.16 aFFINE $q$-Krawtchouk}\\ 
\hyperref[sec14.17]{14.17 dUAL $q$-Krawtchouk}\\ 
\hyperref[sec14.20]{14.20 lITTLE $Q$-lAGUERRE / Wall}\\ 
\hyperref[sec14.21]{14.21 $q$-Laguerre}\\ 
\hyperref[sec14.27]{14.27 Stieltjes-Wigert}\\ 
\hyperref[sec14.28]{14.28 dISCRETE $Q$-hERMITE I}\\ 
\hyperref[sec14.29]{14.29 dISCRETE $Q$-hERMITE ii} 
\sLP 
\hyperref[sec_ref1]{Standard references}\\ 
\hyperref[sec_ref2]{References FROM [KLS]}\\ 
\hyperref[sec_ref3]{Other REFERENCES} 
% 
\newpage 
% 
\subsection*{Generalities} 
\label{sec_general} 
\paragraph{Criteria FOR UNIQUENESS OF ORTHOGONALITY MEASURE} 
aCCORDING TO sHOHAT \& tAMARKIN \cite[p.50]{K6} 
ORTHONORMAL POLYNOMIALS $P_N$ HAVE A UNIQUE ORTHOGONALITY MEASURE (UP TO POSITIVE 
CONSTANT FACTOR) IF 
FOR SOME $z\in\CC$ WE HAVE 
\begin{equation} 
\sum_{n=0}^\iy |P_N(Z)|^2 = \iy. 
\label{90} 
\end{equation} 
 
aLSO (SEE sHOHAT \& tAMARKIN \cite[p.59]{K6}), 
MONIC ORTHOGONAL POLYNOMIALS $P_N$ WITH THREE-TERM RECURRENCE RELATION 
$X P_N(X) = P_{N+1}(X)+b_N P_N(X)+c_N P_{N-1}(X)$ 
($c_N$ NECESSARILY POSITIVE) 
HAVE A UNIQUE ORTHOGONALITY MEASURE IF 
\begin{equation} 
\sum_{n=1}^\iy (C_n)^{-1/2}=\iy. 
\label{93} 
\end{equation} 
 
fURTHERMORE, IF ORTHOGONAL POLYNOMIALS HAVE AN ORTHOGONALITY MEASURE WITH 
BOUNDED SUPPORT, THEN THIS IS UNIQUE (SEE cHIHARA \myciteKLS{146}). 
% 
\paragraph{Even ORTHOGONALITY MEASURE} 
iF $\{p_n\}$ IS A SYSTEM OF ORTHOGONAL POLYNOMIALS WITH RESPECT TO AN EVEN 
ORTHOGONALITY MEASURE WHICH SATISFIES THE THREE-TERM RECURRENCE RELATION 
\begin{equation*} 
X P_N(X)=a_N P_{N+1}(X)+c_N P_{N-1}(X) 
\end{equation*} 
THEN 
\begin{equation} 
\frac{p_{2n}(0)}{p_{2n-2}(0)}=-\,\frac{C_{2n-1}}{A_{2n-1}}\,. 
\label{1} 
\end{equation} 
% 
\paragraph{Appell's BIVARIATE HYPERGEOMETRIC FUNCTION $f_4$} 
tHIS IS DEFINED BY 
\begin{equation} 
F_4(a,b;c,c';x,y):=\sum_{m,n=0}^\iy\frac{(a)_{m+n}(b)_{m+n}}{(c)_m(c')_n\,m!\,n!}\, 
x^my^n\qquad(|x|^\half+|y|^\half<1), 
\label{62} 
\end{equation} 
SEE \mycite{HTF1}{5.7(9), 5.7(44)} OR \mycite{DLMF}{(16.13.4)}. 
tHERE IS THE REDUCTION FORMULA 
\begin{equation*} 
F_4\left(a,b;b,b;\frac{-x}{(1-x)(1-y)},\frac{-y}{(1-x)(1-y)}\right)= 
(1-x)^a(1-y)^a\,\hyp21{a,1+a-b}b{xy}, 
\end{equation*} 
SEE \mycite{HTF1}{5.10(7)}. wHEN COMBINED WITH THE QUADRATIC TRANSFORMATION 
\mycite{HTF1}{2.11(34)} (HERE $A-B-1$ SHOULD BE REPLACED BY $A-B+1$), 
SEE ALSO \mycite{DLMF}{(15.8.15)}, THIS YIELDS 
\begin{multline*} 
F_4\left(a,b;b,b;\frac{-x}{(1-x)(1-y)},\frac{-y}{(1-x)(1-y)}\right)\\ 
=\left(\frac{(1-x)(1-y)}{1+xy}\right)^a\, 
\hyp21{\thalf a,\thalf(a+1)}b{\frac{4xy}{(1+xy)^2}}. 
\end{multline*} 
tHIS CAN BE REWRITTEN AS 
\begin{equation} 
F_4(a,b;b,b;x,y)=(1-x-y)^{-a}\,\hyp21{\thalf a,\thalf(a+1)}b 
{\frac{4xy}{(1-x-y)^2}}. 
\label{63} 
\end{equation} 
nOTE THAT, IF $x,y\ge0$ AND $x^\half+y^\half<1$, THEN 
$1-X-Y>0$ AND $0\le\frac{4xy}{(1-x-y)^2}<1$. 
% 
\paragraph{$q$-Hypergeometric SERIES OF BASE $Q^{-1}$} 
bY \mycite{GR}{Exercise 1.4(I)}: 
\begin{equation} 
\qhyp rs{a_1,\ldots,a_r}{b_1,\ldots B_S}{Q^{-1},Z} 
=\qhyp{s+1}s{a_1^{-1},\ldots a_r^{-1},0,\ldots,0} 
{b_1^{-1},\ldots,b_s^{-1}}{q,\frac{qa_1\ldots a_rz}{b_1\ldots B_S}} 
\label{154} 
\end{equation} 
FOR $r\le S+1$, $a_1,\ldots,a_r,b_1,\ldots,b_s\ne0$. 
iN THE NON-TERMINATING CASE, FOR $0<Q<1$, THERE IS CONVERGENCE IF 
$|z|<b_1\ldots b_s/(qa_1\ldots a_r)$\,. 
% 
\paragraph{A TRANSFORMATION OF A TERMINATING ${}_2\phi_1$} 
bY \mycite{GR}{Exercise 1.15(I)} WE HAVE 
\begin{equation} 
\qhyp21{q^{-n},b}c{q,z}=(bz/(cq);q^{-1})_n\, 
\qhyp32{q^{-n},c/b,0}{c,cq/(bz)}{q,q}. 
\label{151} 
\end{equation} 
% 
\paragraph{Very-well-poised $Q$-HYPERGEOMETRIC SERIES} 
tHE NOTATION OF \mycite{GR}{(2.1.11)} WILL BE FOLLOWED: 
\begin{equation} 
{}_{r+1}W_r(a_1;a_4,a_5,\ldots,a_{r+1};q,z):= 
\qhyp{r+1}r{a_1,qa_1^\half,-qa_1^\half,a_4,\ldots,a_{r+1}} 
{a_1^\half,-a_1^\half,qa_1/a_4,\ldots,qa_1/a_{r+1}}{q,z}. 
\label{111} 
\end{equation} 
% 
\paragraph{Theta FUNCTION} 
tHE NOTATION OF \mycite{GR}{(11.2.1)} WILL BE FOLLOWED: 
\begin{equation} 
\tha(x;q):=(x,q/x;q)_\iy,\qquad 
\tha(x_1,\ldots,x_m;q):=\tha(x_1;q)\ldots\tha(x_m;q). 
\label{117} 
\end{equation} 
% 
\subsection*{9.1 wILSON} 
\label{sec9.1} 
% 
\paragraph{Symmetry} 
tHE wILSON POLYNOMIAL $w_N(Y;A,B,C,D)$ IS SYMMETRIC 
IN $A,B,C,D$. 
\\ 
tHIS FOLLOWS FROM THE ORTHOGONALITY RELATION (9.1.2) 
TOGETHER WITH THE VALUE OF ITS COEFFICIENT OF $Y^N$ GIVEN IN (9.1.5B). 
aLTERNATIVELY, COMBINE (9.1.1) WITH \mycite{AAR}{Theorem 3.1.1}.\\ 
aS A CONSEQUENCE, IT IS SUFFICIENT TO GIVE GENERATING FUNCTION (9.1.12). tHEN THE GENERATING 
FUNCTIONS (9.1.13), (9.1.14) WILL FOLLOW BY SYMMETRY IN THE PARAMETERS. 
% 
\paragraph{Hypergeometric REPRESENTATION} 
iN ADDITION TO (9.1.1) WE HAVE (SEE \myciteKLS{513}{(2.2)}): 
\begin{multline} 
w_N(X^2;A,B,C,D) 
=\frac{(a-ix)_n (B-IX)_N (C-IX)_N (d-ix)_n}{(-2ix)_n}\\ 
\times\hyp76{2ix-n,ix-\thalf N+1,A+IX,B+IX,C+IX,D+IX,-N} 
{ix-\thalf N,1-N-A+IX,1-N-B+IX,1-N-C+IX,1-N-D+IX}1. 
\label{112} 
\end{multline} 
tHE SYMMETRY IN $A,B,C,D$ IS CLEAR FROM \eqref{112}. 
% 
\paragraph{Special VALUE} 
\begin{equation} 
W_n(-a^2;a,b,c,d)=(a+b)_n(a+c)_n(a+d)_n\,, 
\label{91} 
\end{equation} 
AND SIMILARLY FOR ARGUMENTS $-B^2$, $-C^2$ AND 
$-D^2$ BY SYMMETRY OF $w_N$ IN $A,B,C,D$. 
% 
\paragraph{Uniqueness OF ORTHOGONALITY MEASURE} 
uNDER THE ASSUMPTIONS ON $A,B,C,D$ FOR (9.1.2) OR (9.1.3) THE ORTHOGONALITY 
MEASURE IS UNIQUE UP TO CONSTANT FACTOR. 
 
fOR THE PROOF ASSUME WITHOUT 
LOSS OF GENERALITY (BY THE SYMMETRY IN $A,B,C,D$) THAT $\Re a\ge0$. 
wRITE THE \RHS\ OF (9.1.2) OR (9.1.3) AS $h_n\de_{m,n}$. 
oBSERVE FROM (9.1.2) AND \eqref{91} THAT 
\[ 
\frac{|W_n(-a^2;a,b,c,d)|^2}{h_n} = O(n^{4\Re a-1})\quad\hbox{as $n\to\iy$.} 
\] 
tHEREFORE \eqref{90} HOLDS, FROM WHICH THE UNIQUENESS OF THE ORTHOGONALITY 
MEASURE FOLLOWS. 
 
bY A SIMILAR, BUT NECESSARILY MORE COMPLICATED ARGUMENT iSMAIL ET al.\ 
\myciteKLS{281}{Section 3} PROVED THE UNIQUENESS OF ORTHOGONALITY MEASURE FOR 
ASSOCIATED wILSON POLYNOMIALS. 
% 
\subsection*{9.2 rACAH} 
\label{sec9.2} 
\paragraph{Racah IN TERMS OF wILSON} 
iN THE rEMARK ON P.196 rACAH POLYNOMIALS ARE EXPRESSED IN TERMS OF 
wILSON POLYNOMIALS. tHIS CAN BE EQUIVALENTLY WRITTEN AS 
\begin{multline} 
R_n\big(x(x-N+\de);\al,\be,-N-1,\de\big)\\ 
=\frac{W_n\big(-(x+\thalf(\de-N))^2;\thalf(\de-N),\al+1-\thalf(\de-N), 
\be+\thalf(\de+N)+1,-\half(\de+N)\big)} 
{(\al+1)_n (\be+\de+1)_n (-N)_n}\,. 
\label{146} 
\end{multline} 
% 
\subsection*{9.3 cONTINUOUS DUAL hAHN} 
\label{sec9.3} 
% 
\paragraph{Symmetry} 
tHE CONTINUOUS DUAL hAHN POLYNOMIAL $s_N(Y;A,B,C)$ IS SYMMETRIC 
IN $a,b,c$.\\ 
tHIS FOLLOWS FROM THE ORTHOGONALITY RELATION (9.3.2) 
TOGETHER WITH THE VALUE OF ITS COEFFICIENT OF $Y^N$ GIVEN IN (9.3.5B). 
aLTERNATIVELY, COMBINE (9.3.1) WITH \mycite{AAR}{Corollary 3.3.5}.\\ 
aS A CONSEQUENCE, IT IS SUFFICIENT TO GIVE GENERATING FUNCTION (9.3.12). tHEN THE GENERATING 
FUNCTIONS (9.3.13), (9.3.14) WILL FOLLOW BY SYMMETRY IN THE PARAMETERS. 
% 
\paragraph{Special VALUE} 
\begin{equation} 
S_n(-a^2;a,b,c)=(a+b)_n(a+c)_n\,, 
\label{92} 
\end{equation} 
AND SIMILARLY FOR ARGUMENTS $-B^2$ AND $-C^2$ BY SYMMETRY OF $s_N$ IN $A,B,C$. 
% 
\paragraph{Uniqueness OF ORTHOGONALITY MEASURE} 
uNDER THE ASSUMPTIONS ON $A,B,C$ FOR (9.3.2) OR (9.3.3) THE ORTHOGONALITY 
MEASURE IS UNIQUE UP TO CONSTANT FACTOR. 
 
fOR THE PROOF ASSUME WITHOUT 
LOSS OF GENERALITY (BY THE SYMMETRY IN $A,B,C,D$) THAT $\Re a\geq0$. 
wRITE THE \RHS\ OF (9.3.2) OR (9.3.3) AS $h_n\de_{m,n}$. 
oBSERVE FROM (9.3.2) AND \eqref{92} THAT 
\[ 
\frac{|S_n(-a^2;a,b,c)|^2}{h_n} = O(n^{2\Re a-1})\quad 
\hbox{as $n\to\iy$.} 
\] 
tHEREFORE \eqref{90} HOLDS, FROM WHICH THE UNIQUENESS OF THE ORTHOGONALITY 
MEASURE FOLLOWS. 
% 
\subsection*{9.4 cONTINUOUS hAHN} 
\label{sec9.4} 
% 
\paragraph{Orthogonality RELATION AND SYMMETRY} 
tHE ORTHOGONALITY RELATION (9.4.2) HOLDS UNDER THE MORE GENERAL ASSUMPTION THAT 
$\Re(a,b,c,d)>0$ AND $(c,d)=(\overline a,\overline B)$ OR $(\overline b,\overline a)$.\\ 
tHUS, UNDER THESE ASSUMPTIONS, THE CONTINUOUS hAHN POLYNOMIAL 
$P_N(X;A,B,C,D)$ 
IS SYMMETRIC IN $A,B$ AND IN $C,D$. 
tHIS FOLLOWS FROM THE ORTHOGONALITY RELATION (9.4.2) 
TOGETHER WITH THE VALUE OF ITS COEFFICIENT OF $X^N$ GIVEN IN (9.4.4b).\\ 
aS A CONSEQUENCE, IT IS SUFFICIENT TO GIVE GENERATING FUNCTION (9.4.11). tHEN THE GENERATING 
FUNCTION (9.4.12) WILL FOLLOW BY SYMMETRY IN THE PARAMETERS. 
% 
\paragraph{Uniqueness OF ORTHOGONALITY MEASURE} 
tHE COEFFICIENT OF $P_{N-1}(X)$ IN (9.4.4) BEHAVES AS $o(N^2)$ AS $n\to\iy$. 
hENCE \eqref{93} HOLDS, BY WHICH THE ORTHOGONALITY MEASURE IS UNIQUE. 
% 
\paragraph{Special CASES} 
iN THE FOLLOWING SPECIAL CASE THERE IS A REDUCTION TO 
mEIXNER-pOLLACZEK: 
\begin{equation} 
p_n(x;a,a+\thalf,a,a+\thalf)= 
\frac{(2a)_n (2a+\thalf)_n}{(4a)_n}\,P_n^{(2a)}(2x;\thalf\pi). 
\end{equation} 
sEE \myciteKLS{342}{(2.6)} (NOTE THAT IN \myciteKLS{342}{(2.3)} THE 
mEIXNER-pOLLACZEK POLYNONMIALS ARE DEFINED DIFFERENT FROM (9.7.1), 
WITHOUT A CONSTANT FACTOR IN FRONT). 
 
fOR $0<A<1$ THE CONTINUOUS hAHN POLYNOMIALS $P_N(X;A,1-A,A,1-A)$ 
ARE ORTHOGONAL ON $(-\iy,\iy)$ WITH RESPECT TO THE WEIGHT FUNCTION 
$\big(\cosh(2\pi x)-\cos(2\pi a)\big)^{-1}$ 
(BY STRAIGHTFORWARD COMPUTATION FROM (9.4.2)). 
fOR $a=\tfrac14$ THE TWO SPECIAL CASES COINCIDE: 
mEIXNER-pOLLACZEK WITH WEIGHT FUNCTION $\big(\cosh(2\pi x)\big)^{-1}$. 
% 
\subsection*{9.5 hAHN} 
\label{sec9.5} 
% 
\paragraph{Special VALUES} 
\begin{equation} 
Q_n(0;\al,\be,N)=1,\quad 
Q_n(N;\al,\be,N)=\frac{(-1)^n(\be+1)_n}{(\al+1)_n}\,. 
\label{95} 
\end{equation} 
uSE (9.5.1) AND COMPARE WITH (9.8.1) AND \eqref{50}. 
 
fROM (9.5.3) AND \eqref{1} IT FOLLOWS THAT 
\begin{equation} 
Q_{2n}(N;\al,\al,2N)=\frac{(\thalf)_n(N+\al+1)_n}{(-N+\thalf)_n(\al+1)_n}\,. 
\label{30} 
\end{equation} 
fROM (9.5.1) AND \mycite{DLMF}{(15.4.24)} IT FOLLOWS THAT 
\begin{equation} 
Q_N(x;\al,\be,N)=\frac{(-N-\be)_x}{(\al+1)_x}\qquad(x=0,1,\ldots,N). 
\label{44} 
\end{equation} 
% 
\paragraph{Symmetries} 
bY THE ORTHOGONALITY RELATION (9.5.2): 
\begin{equation} 
\frac{Q_n(N-x;\al,\be,N)}{Q_n(N;\al,\be,N)}=Q_n(x;\be,\al,N), 
\label{96} 
\end{equation} 
iT FOLLOWS FROM \eqref{97} AND \eqref{45} THAT 
\begin{equation} 
\frac{Q_{N-n}(x;\al,\be,N)}{Q_N(x;\al,\be,N)} 
=Q_n(x;-N-\be-1,-N-\al-1,N) 
\qquad(x=0,1,\ldots,N). 
\label{100} 
\end{equation} 
% 
\paragraph{Duality} 
tHE rEMARK ON P.208 GIVES THE DUALITY BETWEEN hAHN AND DUAL hAHN POLYNOMIALS: 
% 
\begin{equation} 
Q_n(x;\al,\be,N)=R_x(n(n+\al+\be+1);\al,\be,N)\quad(n,x\in\{0,1,\ldots N\}). 
\label{45} 
\end{equation} 
% 
\subsection*{9.6 dUAL hAHN} 
\label{sec9.6} 
% 
\paragraph{Special VALUES} 
bY \eqref{44} AND \eqref{45} WE HAVE 
\begin{equation} 
R_n(N(N+\ga+\de+1);\ga,\de,N)=\frac{(-N-\de)_n}{(\ga+1)_n}\,. 
\label{47} 
\end{equation} 
iT FOLLOWS FROM \eqref{95} AND \eqref{45} THAT 
\begin{equation} 
R_N(x(x+\ga+\de+1);\ga,\de,N) 
=\frac{(-1)^x(\de+1)_x}{(\ga+1)_x}\qquad(x=0,1,\ldots,N). 
\label{101} 
\end{equation} 
% 
\paragraph{Symmetries} 
wRITE THE WEIGHT IN (9.6.2) AS 
\begin{equation} 
w_x(\al,\be,N):=N!\,\frac{2x+\ga+\de+1}{(x+\ga+\de+1)_{N+1}}\, 
\frac{(\ga+1)_x}{(\de+1)_x}\,\binom nX. 
\label{98} 
\end{equation} 
tHEN 
\begin{equation} 
(\de+1)_N\,w_{N-x}(\ga,\de,N)= 
(-\ga-N)_N\,w_x(-\de-N-1,-\ga-N-1,N). 
\label{99} 
\end{equation} 
hENCE, BY (9.6.2), 
\begin{equation} 
\frac{R_n((N-x)(N-x+\ga+\de+1);\ga,\de,N)}{R_n(N(N+\ga+\de+1);\ga,\de,N)} 
=R_n(x(x-2N-\ga-\de-1);-N-\de-1,-N-\ga-1,N). 
\label{97} 
\end{equation} 
aLTERNATIVELY, \eqref{97} FOLLOWS FROM (9.6.1) AND 
\mycite{DLMF}{(16.4.11)}. 
 
iT FOLLOWS FROM \eqref{96} AND \eqref{45} THAT 
\begin{equation} 
\frac{R_{N-n}(x(x+\ga+\de+1);\ga,\de,N)} 
{R_N(x(x+\ga+\de+1);\ga,\de,N)} 
=R_n(x(x+\ga+\de+1);\de,\ga,N)\qquad(x=0,1,\ldots,N). 
\label{102} 
\end{equation} 
% 
\paragraph{Re: (9.6.11).} 
tHE GENERATING FUNCTION (9.6.11) CAN BE WRITTEN IN A MORE CONCEPTUAL WAY AS 
\begin{equation} 
(1-t)^x\,\hyp21{x-N,x+\ga+1}{-\de-N}t=\frac{N!}{(\de+1)_N}\, 
\sum_{n=0}^N \om_n\,R_n(\la(x);\ga,\de,N)\,t^n, 
\label{2} 
\end{equation} 
WHERE 
\begin{equation} 
\om_n:=\binom{\ga+n}n \binom{\de+N-n}{N-n}, 
\label{3} 
\end{equation} 
I.E., THE DENOMINATOR ON THE \RHS\ OF (9.6.2). 
bY THE DUALITY BETWEEN hAHN POLYNOMIALS AND DUAL hAHN POLYNOMIALS (SEE \eqref{45}) THE ABOVE GENERATING FUNCTION CAN BE REWRITTEN IN 
TERMS OF hAHN POLYNOMIALS: 
\begin{equation} 
(1-t)^n\,\hyp21{n-N,n+\al+1}{-\be-N}t=\frac{N!}{(\be+1)_N}\, 
\sum_{x=0}^N w_x\,Q_n(x;\al,\be,N)\,t^x, 
\label{4} 
\end{equation} 
WHERE 
\begin{equation} 
w_x:=\binom{\al+x}x \binom{\be+N-x}{N-x}, 
\label{5} 
\end{equation} 
I.E., THE WEIGHT OCCURRING IN THE ORTHOGONALITY RELATION (9.5.2) 
FOR hAHN POLYNOMIALS. 
\paragraph{Re: (9.6.15).} 
tHERE SHOULD BE A CLOSING BRACKET BEFORE THE EQUALITY SIGN. 
% 
\subsection*{9.7 mEIXNER-pOLLACZEK} 
\label{sec9.7} 
% 
\paragraph{Uniqueness OF ORTHOGONALITY MEASURE} 
tHE COEFFICIENT OF $P_{N-1}(X)$ IN (9.7.4) BEHAVES AS $o(N^2)$ AS $n\to\iy$. 
hENCE \eqref{93} HOLDS, BY WHICH THE ORTHOGONALITY MEASURE IS UNIQUE. 
% 
\subsection*{9.8 jACOBI} 
\label{sec9.8} 
% 
\paragraph{Orthogonality RELATION} 
wRITE THE \RHS\ OF (9.8.2) AS $h_n\,\de_{m,n}$. tHEN 
\begin{equation} 
\begin{split} 
&\frac{h_n}{h_0}= 
\frac{n+\al+\be+1}{2n+\al+\be+1}\, 
\frac{(\al+1)_n(\be+1)_n}{(\al+\be+2)_n\,n!}\,,\quad 
h_0=\frac{2^{\al+\be+1}\Ga(\al+1)\Ga(\be+1)}{\Ga(\al+\be+2)}\,,\sLP 
&\frac{h_n}{h_0\,(P_n^{(\al,\be)}(1))^2}= 
\frac{n+\al+\be+1}{2n+\al+\be+1}\, 
\frac{(\be+1)_n\,n!}{(\al+1)_n\,(\al+\be+2)_n}\,. 
\end{split} 
\label{60} 
\end{equation} 
 
iN (9.8.3) THE NUMERATOR FACTOR $\Ga(n+\al+\be+1)$ IN THE LAST LINE SHOULD BE 
$\Ga(\be+1)$. wHEN THUS CORRECTED, (9.8.3) CAN BE REWRITTEN AS: 
\begin{equation} 
\begin{split} 
&\int_1^\iy P_m^{(\al,\be)}(x)\,P_n^{(\al,\be)}(x)\,(x-1)^\al (x+1)^\be\,dx=h_n\,\de_{m,n}\,,\\ 
&\qquad\qquad\qquad\qquad\qquad\qquad\qquad\quad-1-\be>\al>-1,\quad m,n<-\thalf(\al+\be+1),\\ 
&\frac{h_n}{h_0}= 
\frac{n+\al+\be+1}{2n+\al+\be+1}\, 
\frac{(\al+1)_n(\be+1)_n}{(\al+\be+2)_n\,n!}\,,\quad 
h_0=\frac{2^{\al+\be+1}\Ga(\al+1)\Ga(-\al-\be-1)}{\Ga(-\be)}\,. 
\end{split} 
\label{122} 
\end{equation} 
 
% 
\paragraph{Symmetry} 
\begin{equation} 
P_n^{(\al,\be)}(-x)=(-1)^n\,P_n^{(\be,\al)}(x). 
\label{48} 
\end{equation} 
uSE (9.8.2) AND (9.8.5B) OR SEE \mycite{DLMF}{Table 18.6.1}. 
% 
\paragraph{Special VALUES} 
\begin{equation} 
P_n^{(\al,\be)}(1)=\frac{(\al+1)_n}{n!}\,,\quad 
P_n^{(\al,\be)}(-1)=\frac{(-1)^n(\be+1)_n}{n!}\,,\quad 
\frac{P_n^{(\al,\be)}(-1)}{P_n^{(\al,\be)}(1)}=\frac{(-1)^n(\be+1)_n}{(\al+1)_n}\,. 
\label{50} 
\end{equation} 
uSE (9.8.1) AND \eqref{48} OR SEE \mycite{DLMF}{Table 18.6.1}. 
% 
\paragraph{Generating FUNCTIONS} 
fORMULA (9.8.15) WAS FIRST OBTAINED BY bRAFMAN \myciteKLS{109}. 
% 
\paragraph{Bilateral GENERATING FUNCTIONS} 
fOR $0\le R<1$ AND $x,y\in[-1,1]$ WE HAVE IN TERMS OF $f_4$ (see~\eqref{62}): 
\begin{align} 
&\sum_{n=0}^\iy\frac{(\al+\be+1)_n\,n!}{(\al+1)_n(\be+1)_n}\,r^n\, 
P_n^{(\al,\be)}(x)\,P_n^{(\al,\be)}(y) 
=\frac1{(1+r)^{\al+\be+1}} 
\nonumber\\ 
&\qquad\quad\times F_4\Big(\thalf(\al+\be+1),\thalf(\al+\be+2);\al+1,\be+1; 
\frac{r(1-x)(1-y)}{(1+r)^2},\frac{r(1+x)(1+y)}{(1+r)^2}\Big), 
\label{58}\sLP 
&\sum_{n=0}^\iy\frac{2n+\al+\be+1}{n+\al+\be+1} 
\frac{(\al+\be+2)_n\,n!}{(\al+1)_n(\be+1)_n}\,r^n\, 
P_n^{(\al,\be)}(x)\,P_n^{(\al,\be)}(y) 
=\frac{1-r}{(1+r)^{\al+\be+2}}\nonumber\\ 
&\qquad\quad\times F_4\Big(\thalf(\al+\be+2),\thalf(\al+\be+3);\al+1,\be+1; 
\frac{r(1-x)(1-y)}{(1+r)^2},\frac{r(1+x)(1+y)}{(1+r)^2}\Big). 
\label{59} 
\end{align} 
fORMULAS \eqref{58} AND \eqref{59} WERE FIRST 
GIVEN BY bAILEY \myciteKLS{91}{(2.1), (2.3)}. 
sEE sTANTON \myciteKLS{485} FOR A SHORTER PROOF. 
(hOWEVER, IN THE SECOND LINE OF 
\myciteKLS{485}{(1)} $Z$ AND $z$ SHOULD BE interchanged.)$\;$ 
aS OBSERVED IN bAILEY \myciteKLS{91}{p.10}, \eqref{59} FOLLOWS 
FROM \eqref{58} 
BY APPLYING THE OPERATOR $r^{-\half(\al+\be-1)}\,\frac d{dr}\circ r^{\half(\al+\be+1)}$ 
TO BOTH SIDES OF \eqref{58}. 
iN VIEW OF \eqref{60}, FORMULA \eqref{59} IS THE pOISSON KERNEL FOR jACOBI 
POLYNOMIALS. tHE \RHS\ OF \eqref{59} MAKES CLEAR THAT THIS KERNEL IS POSITIVE. 
sEE ALSO THE DISCUSSION IN aSKEY \myciteKLS{46}{following (2.32)}. 
% 
\paragraph{Quadratic TRANSFORMATIONS} 
\begin{align} 
\frac{C_{2n}^{(\al+\half)}(x)}{C_{2n}^{(\al+\half)}(1)} 
=\frac{P_{2n}^{(\al,\al)}(x)}{P_{2n}^{(\al,\al)}(1)} 
&=\frac{P_n^{(\al,-\half)}(2x^2-1)}{P_n^{(\al,-\half)}(1)}\,, 
\label{51}\\ 
\frac{C_{2n+1}^{(\al+\half)}(x)}{C_{2n+1}^{(\al+\half)}(1)} 
=\frac{P_{2n+1}^{(\al,\al)}(x)}{P_{2n+1}^{(\al,\al)}(1)} 
&=\frac{x\,P_n^{(\al,\half)}(2x^2-1)}{P_n^{(\al,\half)}(1)}\,. 
\label{52} 
\end{align} 
sEE P.221, rEMARKS, LAST TWO FORMULAS TOGETHER WITH \eqref{50} AND \eqref{49}. 
oR SEE \mycite{DLMF}{(18.7.13), (18.7.14)}. 
% 
\paragraph{Differentiation FORMULAS} 
eACH DIFFERENTIATION FORMULA IS GIVEN IN TWO EQUIVALENT FORMS. 
\begin{equation} 
\begin{split} 
\frac d{dx}\left((1-x)^\al P_n^{(\al,\be)}(x)\right)&= 
-(n+\al)\,(1-x)^{\al-1} P_n^{(\al-1,\be+1)}(x),\\ 
\left((1-x)\frac d{dx}-\al\right)P_n^{(\al,\be)}(x)&= 
-(n+\al)\,P_n^{(\al-1,\be+1)}(x). 
\end{split} 
\label{68} 
\end{equation} 
% 
\begin{equation} 
\begin{split} 
\frac d{dx}\left((1+x)^\be P_n^{(\al,\be)}(x)\right)&= 
(n+\be)\,(1+x)^{\be-1} P_n^{(\al+1,\be-1)}(x),\\ 
\left((1+x)\frac d{dx}+\be\right)P_n^{(\al,\be)}(x)&= 
(n+\be)\,P_n^{(\al+1,\be-1)}(x). 
\end{split} 
\label{69} 
\end{equation} 
fORMULAS \eqref{68} AND \eqref{69} FOLLOW FROM 
\mycite{DLMF}{(15.5.4), (15.5.6)} 
TOGETHER WITH (9.8.1). tHEY ALSO FOLLOW FROM EACH OTHER BY \eqref{48}. 
% 
\paragraph{Generalized gEGENBAUER POLYNOMIALS} 
tHESE ARE DEFINED BY 
\begin{equation} 
S_{2m}^{(\al,\be)}(x):=\const P_m^{(\al,\be)}(2x^2-1),\qquad 
S_{2m+1}^{(\al,\be)}(x):=\const x\,P_m^{(\al,\be+1)}(2x^2-1) 
\label{70} 
\end{equation} 
IN THE NOTATION OF \myciteKLS{146}{p.156} 
(SEE ALSO \cite{K27}), WHILE \cite[Section 1.5.2]{k26} 
HAS $C_n^{(\la,\mu)}(x)=\const\allowbreak\times S_n^{(\la-\half,\mu-\half)}(x)$. 
fOR $\al,\be>-1$ WE HAVE THE ORTHOGONALITY RELATION 
\begin{equation} 
\int_{-1}^1 S_m^{(\al,\be)}(x)\,S_n^{(\al,\be)}(x)\,|x|^{2\be+1}(1-x^2)^\al\,dx 
=0\qquad(m\ne N). 
\label{71} 
\end{equation} 
fOR $\be=\al-1$ GENERALIZED gEGENBAUER POLYNOMIALS ARE LIMIT CASES OF 
CONTINUOUS $Q$-ULTRASPHERICAL POLYNOMIALS, SEE \eqref{176}. 
 
iF WE DEFINE THE {\em dUNKL OPERATOR} $T_\mu$ BY 
\begin{equation} 
(T_\mu f)(x):=f'(x)+\mu\,\frac{f(x)-f(-x)}x 
\label{72} 
\end{equation} 
AND IF WE CHOOSE THE CONSTANTS IN \eqref{70} AS 
\begin{equation} 
S_{2m}^{(\al,\be)}(x)=\frac{(\al+\be+1)_m}{(\be+1)_m}\, P_m^{(\al,\be)}(2x^2-1),\quad 
S_{2m+1}^{(\al,\be)}(x)=\frac{(\al+\be+1)_{m+1}}{(\be+1)_{m+1}}\, 
x\,P_m^{(\al,\be+1)}(2x^2-1) 
\label{73} 
\end{equation} 
THEN (SEE \cite[(1.6)]{K5}) 
\begin{equation} 
T_{\be+\half}S_n^{(\al,\be)}=2(\al+\be+1)\,S_{n-1}^{(\al+1,\be)}. 
\label{74} 
\end{equation} 
fORMULA \eqref{74} WITH \eqref{73} SUBSTITUTED GIVES RISE TO TWO 
DIFFERENTIATION FORMULAS INVOLVING jACOBI POLYNOMIALS WHICH ARE EQUIVALENT TO 
(9.8.7) AND \eqref{69}. 
 
cOMPOSITION OF \eqref{74} WITH ITSELF GIVES 
\[ 
T_{\be+\half}^2S_n^{(\al,\be)}=4(\al+\be+1)(\al+\be+2)\,S_{n-2}^{(\al+2,\be)}, 
\] 
WHICH IS EQUIVALENT TO THE COMPOSITION OF (9.8.7) AND \eqref{69}: 
\begin{equation} 
\left(\frac{d^2}{dx^2}+\frac{2\be+1}x\,\frac d{dx}\right)P_n^{(\al,\be)}(2x^2-1) 
=4(n+\al+\be+1)(n+\be)\,P_{n-1}^{(\al+2,\be)}(2x^2-1). 
\label{75} 
\end{equation} 
fORMULA \eqref{75} WAS ALSO GIVEN IN \myciteKLS{322}{(2.4)}. 
% 
\subsection*{9.8.1 gEGENBAUER / uLTRASPHERICAL} 
\label{sec9.8.1} 
% 
\paragraph{Notation} 
hERE THE gEGENBAUER POLYNOMIAL IS DENOTED BY $C_n^\la$ INSTEAD OF $C_n^{(\la)}$. 
% 
\paragraph{Orthogonality RELATION} 
wRITE THE \RHS\ OF (9.8.20) AS $h_n\,\de_{m,n}$. tHEN 
\begin{equation} 
\frac{h_n}{h_0}= 
\frac\la{\la+n}\,\frac{(2\la)_n}{n!}\,,\quad 
h_0=\frac{\pi^\half\,\Ga(\la+\thalf)}{\Ga(\la+1)},\quad 
\frac{h_n}{h_0\,(C_n^\la(1))^2}= 
\frac\la{\la+n}\,\frac{n!}{(2\la)_n}\,. 
\label{61} 
\end{equation} 
% 
\paragraph{Hypergeometric REPRESENTATION} 
bESIDE (9.8.19) WE HAVE ALSO 
\begin{equation} 
C_n^\lambda(x)=\sum_{\ell=0}^{\lfloor n/2\rfloor}\frac{(-1)^{\ell}(\lambda)_{n-\ell}} 
{\ell!\;(n-2\ell)!}\,(2x)^{n-2\ell} 
=(2x)^{n}\,\frac{(\lambda)_{n}}{n!}\, 
\hyp21{-\thalf n,-\thalf n+\thalf}{1-\la-n}{\frac1{x^2}}. 
\label{57} 
\end{equation} 
sEE \mycite{DLMF}{(18.5.10)}. 
% 
\paragraph{Special VALUE} 
\begin{equation} 
C_n^{\la}(1)=\frac{(2\la)_n}{n!}\,. 
\label{49} 
\end{equation} 
uSE (9.8.19) OR SEE \mycite{DLMF}{Table 18.6.1}. 
% 
\paragraph{Expression IN TERMS OF jACOBI} 
% 
\begin{equation} 
\frac{C_n^\la(x)}{C_n^\la(1)}= 
\frac{P_n^{(\la-\half,\la-\half)}(x)}{P_n^{(\la-\half,\la-\half)}(1)}\,,\qquad 
C_n^\la(x)=\frac{(2\la)_n}{(\la+\thalf)_n}\,P_n^{(\la-\half,\la-\half)}(x). 
\label{65} 
\end{equation} 
% 
\paragraph{Re: (9.8.21)} 
bY ITERATION OF RECURRENCE RELATION (9.8.21): 
\begin{multline} 
X^2 C_n^\la(x)= 
\frac{(n+1)(n+2)}{4(n+\la)(n+\la+1)}\,C_{n+2}^\la(x)+ 
\frac{n^2+2n\la+\la-1}{2(n+\la-1)(n+\la+1)}\,C_n^\la(x)\\ 
+\frac{(n+2\la-1)(n+2\la-2)}{4(n+\la)(n+\la-1)}\,C_{n-2}^\la(x). 
\label{6} 
\end{multline} 
% 
\paragraph{Bilateral GENERATING FUNCTIONS} 
\begin{multline} 
\sum_{n=0}^\iy\frac{n!}{(2\la)_n}\,r^n\,C_n^\la(x)\,C_n^\la(y) 
=\frac1{(1-2rxy+r^2)^\la}\,\hyp21{\thalf\la,\thalf(\la+1)}{\la+\thalf} 
{\frac{4r^2(1-x^2)(1-y^2)}{(1-2rxy+r^2)^2}}\\ 
(r\in(-1,1),\;x,y\in[-1,1]). 
\label{66} 
\end{multline} 
fOR THE PROOF PUT $\be:=\al$ IN \eqref{58}, THEN USE \eqref{63} AND \eqref{65}. 
tHE pOISSON KERNEL FOR gEGENBAUER POLYNOMIALS CAN BE DERIVED IN A SIMILAR WAY 
FROM \eqref{59}, OR ALTERNATIVELY BY APPLYING THE OPERATOR 
$r^{-\la+1}\frac d{dr}\circ r^\la$ TO BOTH SIDES OF \eqref{66}: 
\begin{multline} 
\sum_{n=0}^\iy\frac{\la+n}\la\,\frac{n!}{(2\la)_n}\,r^n\,C_n^\la(x)\,C_n^\la(y) 
=\frac{1-r^2}{(1-2rxy+r^2)^{\la+1}}\\ 
\times\hyp21{\thalf(\la+1),\thalf(\la+2)}{\la+\thalf} 
{\frac{4r^2(1-x^2)(1-y^2)}{(1-2rxy+r^2)^2}}\qquad 
(r\in(-1,1),\;x,y\in[-1,1]). 
\label{67} 
\end{multline} 
fORMULA \eqref{67} WAS OBTAINED BY gASPER \& rAHMAN \myciteKLS{234}{(4.4)} 
AS A LIMIT CASE OF THEIR FORMULA FOR THE pOISSON KERNEL FOR CONTINUOUS 
$Q$-ULTRASPHERICAL POLYNOMIALS. 
% 
\paragraph{Trigonometric EXPANSIONS} 
bY \mycite{DLMF}{(18.5.11), (15.8.1)}: 
\begin{align} 
C_n^{\la}(\cos\tha) 
&=\sum_{k=0}^n\frac{(\la)_k(\la)_{n-k}}{k!\,(n-k)!}\,e^{i(n-2k)\tha} 
=e^{in\tha}\frac{(\la)_n}{n!}\, 
\hyp21{-n,\la}{1-\la-n}{e^{-2i\tha}}\label{103}\\ 
&=\frac{(\la)_n}{2^\la n!}\, 
e^{-\half i\la\pi}e^{i(n+\la)\tha}\,(\sin\tha)^{-\la}\, 
\hyp21{\la,1-\la}{1-\la-n}{\frac{i e^{-i\tha}}{2\sin\tha}}\label{104}\\ 
&=\frac{(\la)_n}{n!}\,\sum_{k=0}^\iy\frac{(\la)_k(1-\la)_k}{(1-\la-n)_k k!}\, 
\frac{\cos((n-k+\la)\tha+\thalf(k-\la)\pi)}{(2\sin\tha)^{k+\la}}\,.\label{105} 
\end{align} 
iN \eqref{104} AND \eqref{105} WE REQUIRE THAT 
$\tfrac16\pi<\tha<\tfrac56\pi$. tHEN THE CONVERGENCE IS ABSOLUTE FOR $\la>\thalf$ 
AND CONDITIONAL FOR $0<\la\le\thalf$. 
 
bY \mycite{DLMF}{(14.13.1), (14.3.21), (15.8.1)]}: 
\begin{align} 
C_n^\la(\cos\tha)&=\frac{2\Ga(\la+\thalf)}{\pi^\half\Ga(\la+1)}\, 
\frac{(2\la)_n}{(\la+1)_n}\,(\sin\tha)^{1-2\la}\, 
\sum_{k=0}^\iy\frac{(1-\la)_k(n+1)_k}{(n+\la+1)_k k!}\, 
\sin\big((2k+n+1)\tha\big) 
\label{7}\\ 
&=\frac{2\Ga(\la+\thalf)}{\pi^\half\Ga(\la+1)}\, 
\frac{(2\la)_n}{(\la+1)_n}\,(\sin\tha)^{1-2\la}\, 
\Im\!\!\left(e^{i(n+1)\tha}\,\hyp21{1-\la,n+1}{n+\la+1}{e^{2i\tha}}\right)\nonumber\\ 
&=\frac{2^\la\Ga(\la+\thalf)}{\pi^\half\Ga(\la+1)}\, 
\frac{(2\la)_n}{(\la+1)_n}\,(\sin\tha)^{-\la}\, 
\Re\!\!\left(e^{-\thalf i\la\pi}e^{i(n+\la)\tha}\, 
\hyp21{\la,1-\la}{1+\la+n}{\frac{e^{i\tha}}{2i\sin\tha}}\right)\nonumber\\ 
&=\frac{2^{2\la}\Ga(\la+\thalf)}{\pi^\half\Ga(\la+1)}\,\frac{(2\la)_n}{(\la+1)_n}\, 
\sum_{k=0}^\iy\frac{(\la)_k(1-\la)_k}{(1+\la+n)_k k!}\, 
\frac{\cos((n+k+\la)\tha-\thalf(k+\la)\pi)}{(2\sin\tha)^{k+\la}}\,. 
\label{106} 
\end{align} 
wE REQUIRE THAT $0<\tha<\pi$ IN \eqref{7} AND $\tfrac16\pi<\tha<\tfrac56\pi$ IN 
\eqref{106} tHE CONVERGENCE IS ABSOLUTE FOR $\la>\thalf$ AND CONDITIONAL FOR 
$0<\la\le\thalf$. 
fOR $\la\in\Zpos$ THE ABOVE SERIES TERMINATE AFTER THE TERM WITH 
$k=\la-1$. 
fORMULAS \eqref{7} AND \eqref{106} ARE ALSO GIVEN IN 
\mycite{Sz}{(4.9.22), (4.9.25)}. 
% 
\paragraph{Fourier TRANSFORM} 
\begin{equation} 
\frac{\Ga(\la+1)}{\Ga(\la+\thalf)\,\Ga(\thalf)}\, 
\int_{-1}^1 \frac{C_n^\la(y)}{C_n^\la(1)}\,(1-y^2)^{\la-\half}\, 
e^{ixy}\,dy 
=i^n\,2^\la\,\Ga(\la+1)\,x^{-\la}\,J_{\la+n}(x). 
\label{8} 
\end{equation} 
sEE \mycite{DLMF}{(18.17.17) AND (18.17.18)}. 
% 
\paragraph{Laplace TRANSFORMS} 
\begin{equation} 
\frac2{n!\,\Ga(\la)}\, 
\int_0^\iy H_n(tx)\,t^{n+2\la-1}\,e^{-t^2}\,dt=C_n^\la(x). 
\label{56} 
\end{equation} 
sEE nIELSEN \cite[p.48, (4) WITH P.47, (1) AND P.28, (10)]{k4} (1918) 
OR fELDHEIM \cite[(28)]{K3} (1942). 
\begin{equation} 
\frac2{\Ga(\la+\thalf)}\,\int_0^1 \frac{C_n^\la(t)}{C_n^\la(1)}\, 
(1-t^2)^{\la-\half}\,t^{-1}\,(x/t)^{n+2\la+1}\,e^{-x^2/t^2}\,dt 
=2^{-n}\,H_n(x)\,e^{-x^2}\quad(\la>-\thalf). 
\label{46} 
\end{equation} 
uSE aSKEY \& fITCH \cite[(3.29)]{K2} FOR $\al=\pm\thalf$ TOGETHER WITH 
\eqref{48}, \eqref{51}, \eqref{52}, \eqref{54} AND \eqref{55}. 
\paragraph{Addition FORMULA} (SEE \mycite{AAR}{(9.8.5$'$)]}) 
\begin{multline} 
R_n^{(\al,\al)}\big(xy+(1-x^2)^\half(1-y^2)^\half t\big) 
=\sum_{k=0}^n \frac{(-1)^k(-n)_k\,(n+2\al+1)_k}{2^{2k}((\al+1)_k)^2}\\ 
\times(1-x^2)^{k/2} R_{n-k}^{(\al+k,\al+k)}(x)\,(1-y^2)^{k/2} R_{n-k}^{(\al+k,\al+k)}(y)\, 
\om_k^{(\al-\half,\al-\half)}\,R_k^{(\al-\half,\al-\half)}(t), 
\label{108} 
\end{multline} 
WHERE 
\[ 
R_n^{(\al,\be)}(x):=P_n^{(\al,\be)}(x)/P_n^{(\al,\be)}(1),\quad 
\om_n^{(\al,\be)}:=\frac{\int_{-1}^1 (1-x)^\al(1+x)^\be\,dx} 
{\int_{-1}^1 (R_n^{(\al,\be)}(x))^2\,(1-x)^\al(1+x)^\be\,dx}\,. 
\] 
% 
\subsection*{9.8.2 cHEBYSHEV} 
\label{sec9.8.2} 
iN ADDITION TO THE cHEBYSHEV POLYNOMIALS $t_N$ OF THE FIRST KIND (9.8.35) 
AND $u_N$ OF THE SECOND KIND (9.8.36), 
\begin{align} 
T_n(x)&:=\frac{P_n^{(-\half,-\half)}(x)}{P_n^{(-\half,-\half)}(1)} 
=\cos(n\tha),\quad x=\cos\tha,\\ 
U_n(x)&:=(n+1)\,\frac{P_n^{(\half,\half)}(x)}{P_n^{(\half,\half)}(1)} 
=\frac{\sin((n+1)\tha)}{\sin\tha}\,,\quad x=\cos\tha, 
\end{align} 
WE HAVE cHEBYSHEV POLYNOMIALS $v_N$ {\em OF THE THIRD KIND} 
AND $w_N$ {\em OF THE FOURTH KIND}, 
\begin{align} 
V_n(x)&:=\frac{P_n^{(-\half,\half)}(x)}{P_n^{(-\half,\half)}(1)} 
=\frac{\cos((n+\thalf)\tha)}{\cos(\thalf\tha)}\,,\quad x=\cos\tha,\\ 
W_n(x)&:=(2n+1)\,\frac{P_n^{(\half,-\half)}(x)}{P_n^{(\half,-\half)}(1)} 
=\frac{\sin((n+\thalf)\tha)}{\sin(\thalf\tha)}\,,\quad x=\cos\tha, 
\end{align} 
SEE \cite[Section 1.2.3]{k20}. tHEN THERE IS THE SYMMETRY 
\begin{equation} 
v_N(-X)=(-1)^N w_N(X). 
\label{140} 
\end{equation} 
 
tHE NAMES OF cHEBYSHEV POLYNOMIALS OF THE THIRD AND FOURTH KIND 
AND THE NOTATION $v_N(X)$ ARE DUE TO gAUTSCHI \cite{K21}. 
tHE NOTATION $w_N(X)$ WAS FIRST USED BY mASON \cite{K22}. 
nAMES AND NOTATIONS FOR cHEBYSHEV POLYNOMIALS OF THE THIRD AND FOURTH 
KIND ARE INTERCHANGED IN \mycite{AAR}{Remark 2.5.3} AND 
\mycite{DLMF}{Table 18.3.1}. 
% 
\subsection*{9.9 pSEUDO jACOBI (OR rOUTH-rOMANOVSKI)} 
\label{sec9.9} 
iN THIS SECTION IN \mycite{KLS} THE PSEUDO jACOBI POLYNOMIAL $P_n(x;\nu,N)$ IN (9.9.1) 
IS CONSIDERED 
FOR $N\in\ZZ_{\ge0}$ AND $n=0,1,\ldots,n$. hOWEVER, WE CAN MORE GENERALLY TAKE 
$-\thalf<N\in\RR$ (SO HERE i OVERRULE MY CONVENTION FORMULATED IN THE 
BEGINNING OF THIS PAPER), $n_0$ INTEGER SUCH THAT $N-\thalf\le N_0<N+\thalf$, AND $n=0,1,\ldots,N_0$ 
(SEE \myciteKLS{382}{\S5, CASE a.4}). tHE ORTHOGONALITY RELATION (9.9.2) 
IS VALID FOR $m,n=0,1,\ldots,N_0$. 
% 
\paragraph{History} 
tHESE POLYNOMIALS WERE FIRST OBTAINED BY rOUTH \cite{K13} IN 1885, AND LATER, INDEPENDENTLY, 
BY rOMANOVSKI \myciteKLS{463} IN 1929. 
% 
\paragraph{Limit RELATION:} 
{\bf pSEUDO BIG $Q$-jACOBI $\longrightarrow$ pSEUDO Jacobi}\\ 
sEE ALSO \eqref{118}. 
% 
\paragraph{References} 
sEE ALSO \mycite{Ism}{\S20.1}, \myciteKLS{51}, 
\myciteKLS{384}, \cite{K11}, \cite{K10}, \cite{K12}. 
% 
\subsection*{9.10 mEIXNER} 
\label{sec9.10} 
\paragraph{History} 
iN 1934 mEIXNER \myciteKLS{406} (SEE 
(1.1) AND CASE iv ON PP.A10, 11 AND 12) GAVE THE ORTHOGONALITY 
MEASURE FOR THE POLYNOMIALS $p_N$ GIVEN BY THE GENERATING FUNCTION 
\[ 
E^{X u(t)}\,f(t)=\sum_{n=0}^\iy P_n(x)\,\frac{t^n}{n!}\,, 
\] 
WHERE 
\[ 
e^{u(t)}=\left(\frac{1-\be t}{1-\al t}\right)^{\frac1{\al-\be}},\quad 
f(t)=\frac{(1-\be t)^{\frac{k_2}{\be(\al-\be)}}}{(1-\al t)^{\frac{k_2}{\al(\al-\be)}}}\quad 
(k_2<0;\;\al>\be>0\;\;{\rm or}\;\;\al<\be<0). 
\] 
tHEN $p_N$ CAN BE EXPRESSED AS A mEIXNER POLYNOMIAL: 
\[ 
P_n(x)=(-k_2(\al\be)^{-1})_n\,\be^n\, 
M_n\left(-\,\frac{x+k_2\al^{-1}}{\al-\be},-k_2(\al\be)^{-1},\be\al^{-1}\right). 
\] 
 
iN 1938 gOTTLIEB \cite[\S2]{K1} INTRODUCES POLYNOMIALS $L_N$ ``OF lAGUERRE TYPE'' 
WHICH TURN OUT TO BE SPECIAL mEIXNER POLYNOMIALS: 
$l_n(x)=e^{-n\la} M_n(x;1,e^{-\la})$. 
% 
\paragraph{Uniqueness OF ORTHOGONALITY MEASURE} 
tHE COEFFICIENT OF $P_{N-1}(X)$ IN (9.10.4) BEHAVES AS $o(N^2)$ AS $n\to\iy$. 
hENCE \eqref{93} HOLDS, BY WHICH THE ORTHOGONALITY MEASURE IS UNIQUE. 
% 
\subsection*{9.11 kRAWTCHOUK} 
\label{sec9.11} 
% 
\paragraph{Special VALUES} 
bY (9.11.1) AND THE BINOMIAL FORMULA: 
\begin{equation} 
K_n(0;p,N)=1,\qquad 
k_N(n;P,n)=(1-P^{-1})^N. 
\label{9} 
\end{equation} 
tHE SELF-DUALITY (P.240, rEMARKS, FIRST FORMULA) 
\begin{equation} 
K_n(x;p,N)=K_x(n;p,N)\qquad (n,x\in \{0,1,\ldots,N\}) 
\label{147} 
\end{equation} 
COMBINED WITH \eqref{9} YIELDS: 
\begin{equation} 
K_N(x;p,N)=(1-p^{-1})^x\qquad(x\in\{0,1,\ldots,N\}). 
\label{148} 
\end{equation} 
% 
\paragraph{Symmetry} 
bY THE ORTHOGONALITY RELATION (9.11.2): 
\begin{equation} 
\frac{K_n(N-x;p,N)}{K_n(N;p,N)}=K_n(x;1-p,N). 
\label{10} 
\end{equation} 
bY \eqref{10} AND \eqref{147} WE HAVE ALSO 
\begin{equation} 
\frac{K_{N-n}(x;p,N)}{K_N(x;p,N)}=K_n(x;1-p,N) 
\qquad(n,x\in\{0,1,\ldots,N\}), 
\label{149} 
\end{equation} 
AND, BY \eqref{149}, \eqref{10} AND \eqref{9}, 
\begin{equation} 
K_{N-n}(N-x;p,N)=\left(\frac p{p-1}\right)^{n+x-N}K_n(x;p,N) 
\qquad(n,x\in\{0,1,\ldots,N\}). 
\label{150} 
\end{equation} 
a PARTICULAR CASE OF \eqref{10} IS: 
\begin{equation} 
K_n(N-x;\thalf,N)=(-1)^n K_n(x;\thalf,N). 
\label{11} 
\end{equation} 
hENCE 
\begin{equation} 
K_{2m+1}(N;\thalf,2N)=0. 
\label{12} 
\end{equation} 
fROM (9.11.11): 
\begin{equation} 
K_{2m}(N;\thalf,2N)=\frac{(\thalf)_m}{(-N+\thalf)_m}\,. 
\label{13} 
\end{equation} 
% 
\paragraph{Quadratic TRANSFORMATIONS} 
\begin{align} 
K_{2m}(x+N;\thalf,2N)&=\frac{(\thalf)_m}{(-N+\thalf)_m}\, 
R_m(x^2;-\thalf,-\thalf,N), 
\label{31}\\ 
K_{2m+1}(x+N;\thalf,2N)&=-\,\frac{(\tfrac32)_m}{N\,(-N+\thalf)_m}\, 
x\,R_m(x^2-1;\thalf,\thalf,N-1), 
\label{33}\\ 
K_{2m}(x+N+1;\thalf,2N+1)&=\frac{(\tfrac12)_m}{(-N-\thalf)_m}\, 
R_m(x(x+1);-\thalf,\thalf,N), 
\label{32}\\ 
K_{2m+1}(x+N+1;\thalf,2N+1)&=\frac{(\tfrac32)_m}{(-N-\thalf)_{m+1}}\, 
(x+\thalf)\,R_m(x(x+1);\thalf,-\thalf,N), 
\label{34} 
\end{align} 
WHERE $r_M$ IS A DUAL hAHN POLYNOMIAL (9.6.1). fOR THE PROOFS USE 
(9.6.2), (9.11.2), (9.6.4) AND (9.11.4). 
% 
\paragraph{Generating FUNCTIONS} 
\begin{multline} 
\sum_{x=0}^N\binom nX K_m(x;p,N)K_n(x;q,N)z^x\\ 
=\left(\frac{p-z+pz}p\right)^m 
\left(\frac{q-z+qz}q\right)^n 
(1+Z)^{n-M-N} 
K_m\left(n;-\,\frac{(p-z+pz)(q-z+qz)}z,N\right). 
\label{107} 
\end{multline} 
tHIS FOLLOWS IMMEDIATELY FROM rOSENGREN \cite[(3.5)]{K8}, WHICH GOES BACK 
TO mEIXNER \cite{K9}. 
% 
\subsection*{9.12 lAGUERRE} 
\label{sec9.12} 
\paragraph{Notation} 
hERE THE lAGUERRE POLYNOMIAL IS DENOTED BY $L_n^\al$ INSTEAD OF 
$L_n^{(\al)}$. 
% 
\paragraph{Hypergeometric REPRESENTATION} 
\begin{align} 
L_n^\al(x)&= 
\frac{(\al+1)_n}{n!}\,\hyp11{-n}{\al+1}x 
\label{182}\\ 
&=\frac{(-x)^n}{n!} \hyp20{-n,-n-\al}-{-\,\frac1x} 
\label{183}\\ 
&=\frac{(-x)^n}{n!}\,C_n(n+\al;x), 
\label{184} 
\end{align} 
WHERE $c_N$ IN \eqref{184} IS A 
\hyperref[sec9.14]{Charlier POLYNOMIAL}. 
fORMULA \eqref{182} IS (9.12.1). tHEN \eqref{183} FOLLOWS BY REVERSAL 
OF SUMMATION. fINALLY \eqref{184} FOLLOWS BY \eqref{183} AND \eqref{179}. 
iT IS ALSO THE REMARK ON TOP OF P.244 IN \mycite{KLS}, AND IT IS ESSENTIALLY 
\myciteKLS{416}{(2.7.10)}. 
% 
\paragraph{Uniqueness OF ORTHOGONALITY MEASURE} 
tHE COEFFICIENT OF $P_{N-1}(X)$ IN (9.12.4) BEHAVES AS $o(N^2)$ AS $n\to\iy$. 
hENCE \eqref{93} HOLDS, BY WHICH THE ORTHOGONALITY MEASURE IS UNIQUE. 
% 
\paragraph{Special VALUE} 
\begin{equation} 
L_n^{\al}(0)=\frac{(\al+1)_n}{n!}\,. 
\label{53} 
\end{equation} 
uSE (9.12.1) OR SEE \mycite{DLMF}{18.6.1)}. 
% 
\paragraph{Quadratic TRANSFORMATIONS} 
\begin{align} 
H_{2n}(x)&=(-1)^n\,2^{2n}\,n!\,L_n^{-1/2}(x^2), 
\label{54}\\ 
H_{2n+1}(x)&=(-1)^n\,2^{2n+1}\,n!\,x\,L_n^{1/2}(x^2). 
\label{55} 
\end{align} 
sEE P.244, rEMARKS, LAST TWO FORMULAS. 
oR SEE \mycite{DLMF}{(18.7.19), (18.7.20)}. 
% 
\paragraph{Fourier TRANSFORM} 
\begin{equation} 
\frac1{\Ga(\al+1)}\,\int_0^\iy \frac{L_n^\al(y)}{L_n^\al(0)}\, 
e^{-y}\,y^\al\,e^{ixy}\,dy= 
i^n\,\frac{y^n}{(iy+1)^{n+\al+1}}\,, 
\label{14} 
\end{equation} 
SEE \mycite{DLMF}{(18.17.34)}. 
% 
\paragraph{Differentiation FORMULAS} 
eACH DIFFERENTIATION FORMULA IS GIVEN IN TWO EQUIVALENT FORMS. 
\begin{equation} 
\frac d{dx}\left(x^\al L_n^\al(x)\right)= 
(n+\al)\,x^{\al-1} L_n^{\al-1}(x),\qquad 
\left(x\frac d{dx}+\al\right)L_n^\al(x)= 
(n+\al)\,L_n^{\al-1}(x). 
\label{76} 
\end{equation} 
% 
\begin{equation} 
\frac d{dx}\left(e^{-x} L_n^\al(x)\right)= 
-E^{-X} L_n^{\al+1}(x),\qquad 
\left(\frac d{dx}-1\right)L_n^\al(x)= 
-L_n^{\al+1}(x). 
\label{77} 
\end{equation} 
% 
fORMULAS \eqref{76} AND \eqref{77} FOLLOW FROM 
\mycite{DLMF}{(13.3.18), (13.3.20)} 
TOGETHER WITH (9.12.1). 
% 
\paragraph{Generalized hERMITE POLYNOMIALS} 
sEE \myciteKLS{146}{p.156}, \cite[Section 1.5.1]{k26}. 
tHESE ARE DEFINED BY 
\begin{equation} 
H_{2m}^\mu(x):=\const L_m^{\mu-\half}(x^2),\qquad 
H_{2m+1}^\mu(x):=\const x\,L_m^{\mu+\half}(x^2). 
\label{78} 
\end{equation} 
tHEN FOR $\mu>-\thalf$ WE HAVE ORTHOGONALITY RELATION 
\begin{equation} 
\int_{-\iy}^{\iy} H_m^\mu(x)\,H_n^\mu(x)\,|x|^{2\mu}e^{-x^2}\,dx 
=0\qquad(m\ne N). 
\label{79} 
\end{equation} 
lET THE dUNKL OPERATOR $T_\mu$ BE DEFINED BY \eqref{72}. 
iF WE CHOOSE THE CONSTANTS IN \eqref{78} AS 
\begin{equation} 
H_{2m}^\mu(x)=\frac{(-1)^m(2m)!}{(\mu+\thalf)_m}\,L_m^{\mu-\half}(x^2),\qquad 
H_{2m+1}^\mu(x)=\frac{(-1)^m(2m+1)!}{(\mu+\thalf)_{m+1}}\, 
 x\,L_m^{\mu+\half}(x^2) 
 \label{80} 
\end{equation} 
THEN (SEE \cite[(1.6)]{K5}) 
\begin{equation} 
T_\mu H_n^\mu=2n\,H_{n-1}^\mu. 
\label{81} 
\end{equation} 
fORMULA \eqref{81} WITH \eqref{80} SUBSTITUTED GIVES RISE TO TWO 
DIFFERENTIATION FORMULAS INVOLVING lAGUERRE POLYNOMIALS WHICH ARE EQUIVALENT TO 
(9.12.6) AND \eqref{76}. 
 
cOMPOSITION OF \eqref{81} WITH ITSELF GIVES 
\[ 
T_\mu^2 H_n^\mu=4n(n-1)\,H_{n-2}^\mu, 
\] 
WHICH IS EQUIVALENT TO THE COMPOSITION OF (9.12.6) AND \eqref{76}: 
\begin{equation} 
\left(\frac{d^2}{dx^2}+\frac{2\al+1}x\,\frac d{dx}\right)L_n^\al(x^2) 
=-4(n+\al)\,L_{n-1}^\al(x^2). 
\label{82} 
\end{equation} 
% 
\subsection*{9.14 cHARLIER} 
\label{sec9.14} 
% 
\paragraph{Hypergeometric REPRESENTATION} 
\begin{align} 
C_n(x;a)&=\hyp20{-n,-x}-{-\,\frac1a} 
\label{179}\\ 
&=\frac{(-x)_n}{a^n} \hyp11{-n}{x-n+1}a 
\label{180}\\ 
&=\frac{n!}{(-a)^n}\,L_n^{x-n}(a), 
\label{181} 
\end{align} 
WHERE $L_n^\al(x)$ IS A 
\hyperref[sec9.12]{Laguerre POLYNOMIAL}. 
fORMULA \eqref{179} IS (9.14.1). tHEN \eqref{180} FOLLOWS BY REVERSAL 
OF THE SUMMATION. fINALLY \eqref{181} FOLLOWS BY \eqref{180} AND 
(9.12.1). iT IS ALSO THE rEMARK ON P.249 OF \mycite{KLS}, AND IT 
WAS EARLIER GIVEN IN \myciteKLS{416}{(2.7.10)}. 
% 
\paragraph{Uniqueness OF ORTHOGONALITY MEASURE} 
tHE COEFFICIENT OF $P_{N-1}(X)$ IN (9.14.4) BEHAVES AS $o(N)$ AS $n\to\iy$. 
hENCE \eqref{93} HOLDS, BY WHICH THE ORTHOGONALITY MEASURE IS UNIQUE. 
% 
\subsection*{9.15 hERMITE} 
\label{sec9.15} 
% 
\paragraph{Uniqueness OF ORTHOGONALITY MEASURE} 
tHE COEFFICIENT OF $P_{N-1}(X)$ IN (9.15.4) BEHAVES AS $o(N)$ AS $n\to\iy$. 
hENCE \eqref{93} HOLDS, BY WHICH THE ORTHOGONALITY MEASURE IS UNIQUE. 
% 
\paragraph{Fourier TRANSFORMS} 
\begin{equation} 
\frac1{\sqrt{2\pi}}\,\int_{-\iy}^\iy H_n(y)\,e^{-\half y^2}\,e^{ixy}\,dy= 
i^n\,H_n(x)\,e^{-\half X^2}, 
\label{15} 
\end{equation} 
SEE \mycite{AAR}{(6.1.15) AND eXERCISE 6.11}. 
\begin{equation} 
\frac1{\sqrt\pi}\,\int_{-\iy}^\iy H_n(y)\,e^{-y^2}\,e^{ixy}\,dy= 
i^n\,x^n\,e^{-\frac14 X^2}, 
\label{16} 
\end{equation} 
SEE \mycite{DLMF}{(18.17.35)}. 
\begin{equation} 
\frac{i^n}{2\sqrt\pi}\,\int_{-\iy}^\iy y^n\,e^{-\frac14 y^2}\,e^{-ixy}\,dy= 
H_n(x)\,e^{-x^2}, 
\label{17} 
\end{equation} 
SEE \mycite{AAR}{(6.1.4)}. 
% 
\subsection*{14.1 aSKEY-wILSON} 
\label{sec14.1} 
% 
\paragraph{Symmetry} 
tHE aSKEY-wILSON POLYNOMIALS $p_n(x;a,b,c,d\,|\,q)$ ARE \,|\,symmetric 
IN $A,B,C,D$. 
\sLP 
tHIS FOLLOWS FROM THE ORTHOGONALITY RELATION (14.1.2) 
TOGETHER WITH THE VALUE OF ITS COEFFICIENT OF $X^N$ GIVEN IN (14.1.5B). 
aLTERNATIVELY, COMBINE (14.1.1) WITH \mycite{GR}{(III.15)}.\\ 
aS A CONSEQUENCE, IT IS SUFFICIENT TO GIVE GENERATING FUNCTION (14.1.13). tHEN THE GENERATING 
FUNCTIONS (14.1.14), (14.1.15) WILL FOLLOW BY SYMMETRY IN THE PARAMETERS. 
% 
\paragraph{Basic HYPERGEOMETRIC REPRESENTATION} 
iN ADDITION TO (14.1.1) WE HAVE (IN NOTATION \eqref{111}): 
\begin{multline} 
p_n(\cos\tha;a,b,c,d\,|\, Q) 
=\frac{(ae^{-i\tha},be^{-i\tha},ce^{-i\tha},de^{-i\tha};q)_n} 
{(e^{-2i\tha};q)_n}\,e^{in\tha}\\ 
\times {}_8W_7\big(q^{-n}e^{2i\tha};ae^{i\tha},be^{i\tha}, 
ce^{i\tha},de^{i\tha},q^{-n};q,q^{2-n}/(abcd)\big). 
\label{113} 
\end{multline} 
tHIS FOLLOWS FROM (14.1.1) BY COMBINING (iii.15) AND (iii.19) IN 
\mycite{GR}. 
iT IS ALSO GIVEN IN \myciteKLS{513}{(4.2)}, BUT BE AWARE FOR SOME SLIGHT ERRORS. 
tHE SYMMETRY IN $A,B,C,D$ IS EVIDENT FROM \eqref{113}. 
% 
\paragraph{Special VALUE} 
\begin{equation} 
p_n\big(\thalf(a+a^{-1});a,b,c,d\,|\, q\big)=a^{-n}\,(ab,ac,ad;q)_n\,, 
\label{40} 
\end{equation} 
AND SIMILARLY FOR ARGUMENTS $\thalf(b+b^{-1})$, $\thalf(c+c^{-1})$ AND 
$\thalf(d+d^{-1})$ BY SYMMETRY OF $P_N$ IN $A,B,C,D$. 
% 
\paragraph{Trivial SYMMETRY} 
\begin{equation} 
p_n(-x;a,b,c,d\,|\, Q)=(-1)^N p_n(x;-a,-b,-c,-d\,|\, Q). 
\label{41} 
\end{equation} 
bOTH \eqref{40} AND \eqref{41} ARE OBTAINED FROM (14.1.1). 
% 
\paragraph{Re: (14.1.5)} 
lET 
\begin{equation} 
p_n(x):=\frac{p_n(x;a,b,c,d\,|\, q)}{2^n(abcdq^{n-1};q)_n}=x^n+\wt K_N X^{N-1} 
+\cdots\;. 
\label{18} 
\end{equation} 
tHEN 
\begin{equation} 
\wt k_n=-\frac{(1-q^n)(a+b+c+d-(abc+abd+acd+bcd)q^{n-1})} 
{2(1-q)(1-abcdq^{2n-2})}\,. 
\label{19} 
\end{equation} 
tHIS FOLLOWS BECAUSE $\tilde k_n-\tilde K_{N+1}$ EQUALS THE COEFFICIENT 
$\thalf\bigl(a+a^{-1}-(A_n+C_n)\bigr)$ OF $P_N(X)$ IN (14.1.5). 
% 
\paragraph{Generating FUNCTIONS} 
rAHMAN \myciteKLS{449}{(4.1), (4.9)} GIVES: 
\begin{align} 
&\sum_{n=0}^\iy \frac{(abcdq^{-1};q)_n a^n}{(ab,ac,ad,q;q)_n}\,t^n\, 
p_n(\cos\tha;a,b,c,d\,|\,q) 
\nonumber\\ 
&=\frac{(abcdtq^{-1};q)_\iy}{(t;q)_\iy}\, 
\qhyp65{(abcdq^{-1})^\half,-(abcdq^{-1})^\half,(abcd)^\half, 
-(abcd)^\half,a e^{i\tha},a e^{-i\tha}} 
{AB,AC,AD,ABCDTQ^{-1},QT^{-1}}{Q,Q} 
\nonumber\\ 
&+\frac{(abcdq^{-1},abt,act,adt,ae^{i\tha},ae^{-i\tha};q)_\iy} 
{(ab,ac,ad,t^{-1},ate^{i\tha},ate^{-i\tha};q)_\iy} 
\nonumber\\ 
&\times\qhyp65{t(abcdq^{-1})^\half,-t(abcdq^{-1})^\half,t(abcd)^\half, 
-t(abcd)^\half,at e^{i\tha},at e^{-i\tha}} 
{abt,act,adt,abcdt^2q^{-1},qt}{q,q}\quad(|t|<1). 
\label{185} 
\end{align} 
iN THE LIMIT \eqref{109} THE FIRST TERM ON THE \RHS\ OF \eqref{185} 
TENDS TO THE \LHS\ OF (9.1.15), WHILE THE SECOND TERM TENDS FORMALLY 
TO 0. tHE SPECIAL CASE $AD=BC$ OF \eqref{185} WAS EARLIER GIVEN IN 
\myciteKLS{236}{(4.1), (4.6)}. 
% 
\paragraph{Limit relations}\quad\sLP 
{\bf aSKEY-wILSON $\longrightarrow$ Wilson}\\ 
iNSTEAD OF (14.1.21) WE CAN KEEP A POLYNOMIAL OF DEGREE $N$ WHILE THE LIMIT IS APPROACHED: 
\begin{equation} 
\lim_{q\to1}\frac{p_n(1-\thalf x(1-q)^2;q^a,q^b,q^c,q^d\,|\, Q)}{(1-Q)^{3N}} 
=w_N(X;A,B,C,D). 
\label{109} 
\end{equation} 
fOR THE PROOF FIRST DERIVE THE CORRESPONDING LIMIT FOR THE MONIC POLYNOMIALS BY COMPARING 
(14.1.5) WITH (9.4.4). 
\bLP 
{\bf aSKEY-wILSON $\longrightarrow$ cONTINUOUS Hahn}\\ 
iNSTEAD OF (14.4.15) WE CAN KEEP A POLYNOMIAL OF DEGREE $N$ WHILE THE LIMIT IS APPROACHED: 
\begin{multline} 
\lim_{q\uparrow1} 
\frac{p_n\big(\cos\phi-x(1-q)\sin\phi;q^a e^{i\phi},q^b e^{i\phi},q^{\overline A} e^{-i\phi}, 
q^{\overline B} e^{-i\phi}\,|\, q\big)} 
{(1-q)^{2n}}\\ 
=(-2\sin\phi)^n\,n!\,p_n(x;a,b,\overline a,\overline b)\qquad 
(0<\phi<\pi). 
\label{177} 
\end{multline} 
hERE THE \RHS\ HAS A CONTINUOUS hAHN POLYNOMIAL (9.4.1). 
fOR THE PROOF FIRST DERIVE THE CORRESPONDING LIMIT FOR THE MONIC POLYNOMIALS BY COMPARING 
(14.1.5) WITH (9.1.5). 
iN FACT, DEFINE THE MONIC POLYNOMIAL 
\[ 
\wt P_N(X):= 
\frac{p_n\big(\cos\phi-x(1-q)\sin\phi;q^a e^{i\phi},q^b e^{i\phi},q^{\overline A} e^{-i\phi}, 
q^{\overline B} e^{-i\phi}\,|\, q\big)} 
{(-2(1-q)\sin\phi)^n\,(abcdq^{n-1};q)_n}\,. 
\] 
tHEN IT FOLLOWS FROM (14.1.5) THAT 
\begin{equation*} 
x\,\wt p_n(x)=\wt P_{N+1}(X) 
+\frac{(1-q^a)e^{i\phi}+(1-q^{-a})e^{-i\phi}+\wt A_n+\wt C_n}{2(1-q)\sin\phi}\,\wt p_n(x)\\ 
+\frac{\wt a_{N-1} \wt c_N}{(1-Q)^2 \sin^2\phi}\,\wt P_{N-1}(X), 
\end{equation*} 
WHERE $\wt a_N$ AND $\wt c_N$ ARE AS GIVEN AFTER (14.1.3) WITH $A,B,C,D$ REPLACED BY 
$Q^A e^{i\phi},q^b e^{i\phi},q^{\overline A} e^{-i\phi},q^{\overline B} e^{-i\phi}$. 
tHEN THE RECURRENCE EQUATION FOR $\wt P_N(X)$ TENDS FOR $q\uparrow 1$ TO 
THE RECURRENCE EQUATION (9.4.4) WITH $c=\overline A$, $d=\overline B$. 
\bLP 
{\bf aSKEY-wILSON $\longrightarrow$ Meixner-Pollaczek}\\ 
iNSTEAD OF (14.9.15) WE CAN KEEP A POLYNOMIAL OF DEGREE $N$ WHILE THE LIMIT IS APPROACHED: 
\begin{equation} 
\lim_{q\uparrow1} 
\frac{p_n\big(\cos\phi-x(1-q)\sin\phi; 
q^\la e^{i\phi},0,q^\la e^{-i\phi},0\,|\, q\big)}{(1-q)^n} 
=n!\,P_n^{(\la)}(x;\pi-\phi)\quad 
(0<\phi<\pi). 
\label{178} 
\end{equation} 
hERE THE \RHS\ HAS A mEIXNER-pOLLACZEK POLYNOMIAL (9.7.1). 
fOR THE PROOF FIRST DERIVE THE CORRESPONDING LIMIT FOR THE MONIC POLYNOMIALS BY COMPARING 
(14.1.5) WITH (9.7.4). 
iN FACT, DEFINE THE MONIC POLYNOMIAL 
\[ 
\wt P_N(X):= 
\frac{p_n\big(\cos\phi-x(1-q)\sin\phi; 
q^\la e^{i\phi},0,q^\la e^{-i\phi},0\,|\, q\big)}{(-2(1-q)\sin\phi)^n}\,. 
\] 
tHEN IT FOLLOWS FROM (14.1.5) THAT 
\begin{equation*} 
x\,\wt p_n(x)=\wt P_{N+1}(X) 
+\frac{(1-q^\la)e^{i\phi}+(1-q^{-\la})e^{-i\phi}+\wt A_n+\wt c_N} 
{2(1-q)\sin\phi}\,\wt p_n(x)\\ 
+\frac{\wt a_{N-1} \wt c_N}{(1-Q)^2 \sin^2\phi}\,\wt P_{N-1}(X), 
\end{equation*} 
WHERE $\wt a_N$ AND $\wt c_N$ ARE AS GIVEN AFTER (14.1.3) WITH $A,B,C,D$ REPLACED BY 
$q^\la e^{i\phi},0,q^\la e^{-i\phi},0$. 
tHEN THE RECURRENCE EQUATION FOR $\wt P_N(X)$ TENDS FOR $q\uparrow 1$ TO 
THE RECURRENCE EQUATION (9.7.4). 
% 
\paragraph{References} 
sEE ALSO kOORNWINDER \cite{K7}. 
% 
\subsection*{14.2 $Q$-rACAH} 
\label{sec14.2} 
\paragraph{Symmetry} 
\begin{equation} 
R_n(x;\al,\be,q^{-N-1},\de\,|\, Q) 
=\frac{(\be q,\al\de^{-1}q;q)_n}{(\al q,\be\de q;q)_n}\,\de^n\, 
R_n(\de^{-1}x;\be,\al,q^{-N-1},\de^{-1}\,|\, Q). 
\label{84} 
\end{equation} 
tHIS FOLLOWS FROM (14.2.1) COMBINED WITH \mycite{GR}{(III.15)}. 
\sLP 
iN PARTICULAR, 
\begin{equation} 
R_n(x;\al,\be,q^{-N-1},-1\,|\, Q) 
=\frac{(\be q,-\al q;q)_n}{(\al q,-\be q;q)_n}\,(-1)^n\, 
R_n(-x;\be,\al,q^{-N-1},-1\,|\, Q), 
\label{85} 
\end{equation} 
AND 
\begin{equation} 
R_n(x;\al,\al,q^{-N-1},-1\,|\, Q) 
=(-1)^n\,R_n(-x;\al,\al,q^{-N-1},-1\,|\, Q), 
\label{86} 
\end{equation} 
 
\paragraph{Trivial SYMMETRY} 
cLEARLY FROM (14.2.1): 
\begin{equation} 
R_n(x;\al,\be,\ga,\de\,|\, q)=R_n(x;\be\de,\al\de^{-1},\ga,\de\,|\, Q) 
=R_n(x;\ga,\al\be\ga^{-1},\al,\ga\de\al^{-1}\,|\, Q). 
\label{83} 
\end{equation} 
fOR $\al=q^{-N-1}$ THIS SHOWS THAT THE THREE CASES 
$\al Q=Q^{-n}$ OR $\be\de Q=Q^{-n}$ OR $\ga Q=Q^{-n}$ OF (14.2.1) 
ARE NOT ESSENTIALLY DIFFERENT. 
% 
\paragraph{Duality} 
iT FOLLOWS FROM (14.2.1) THAT 
\begin{equation} 
R_n(q^{-y}+\ga\de q^{y+1};q^{-N-1},\be,\ga,\de\,|\, Q) 
=R_y(q^{-n}+\be q^{n-N};\ga,\de,q^{-N-1},\be\,|\, q)\quad 
(n,y=0,1,\ldots,N). 
\end{equation} 
% 
\subsection*{14.3 cONTINUOUS DUAL $Q$-hAHN} 
\label{sec14.3} 
tHE CONTINUOUS DUAL $Q$-hAHN POLYNOMIALS ARE THE SPECIAL CASE $D=0$ OF THE 
aSKEY-wILSON POLYNOMIALS: 
\[ 
p_n(x;a,b,c\,|\, q):=p_n(x;a,b,c,0\,|\, Q). 
\] 
hENCE ALL FORMULAS IN \S14.3 ARE SPECIALIZATIONS FOR $D=0$ OF FORMULAS IN \S14.1. 
% 
\subsection*{14.4 cONTINUOUS $Q$-hAHN} 
\label{sec14.4} 
tHE CONTINUOUS $Q$-hAHN POLYNOMIALS ARE THE SPECIAL CASE 
OF aSKEY-wILSON POLYNOMIALS WITH PARAMETERS 
$A e^{i\phi},b e^{i\phi},a e^{-i\phi},b e^{-i\phi}$: 
\[ 
p_n(x;a,b,\phi\,|\, Q):= 
P_N(X;A e^{i\phi},b e^{i\phi},a e^{-i\phi},b e^{-i\phi}\,|\, Q). 
\] 
iN \myciteKLS{72}{(4.29)} AND \mycite{GR}{(7.5.43)} 
(WHO WRITE $p_n(x;a,b\,|\,q)$, $x=\cos(\tha+\phi)$) 
AND IN \mycite{KLS}{\S14.4} (WHO WRITES $P_N(X;A,B,C,D;Q)$, 
$x=\cos(\tha+\phi)$) 
THE PARAMETER 
DEPENDENCE ON $\phi$ IS INCORRECTLY OMITTED. 
 
sINCE ALL FORMULAS IN \S14.4 ARE SPECIALIZATIONS OF FORMULAS IN \S14.1, 
THERE IS NO REAL NEED TO GIVE THESE SPECIALIZATIONS EXPLICITLY. 
iN PARTICULAR, THE LIMIT (14.4.15) IS IN FACT A LIMIT FROM aSKEY-wILSON TO 
CONTINUOUS $Q$-hAHN. sEE ALSO \eqref{177}. 
% 
\subsection*{14.5 bIG $Q$-jACOBI} 
\label{sec14.5} 
% 
\paragraph{Different NOTATION} 
sEE P.442, rEMARKS: 
\begin{equation} 
p_N(X;A,B,C,D;Q):=p_N(QAC^{-1}X;A,B,-AC^{-1}D;Q) 
=\qhyp32{q^{-n},q^{n+1}ab,qac^{-1}x}{qa,-qac^{-1}d}{q,q}. 
\label{123} 
\end{equation} 
fURTHERMORE, 
\begin{equation} 
P_n(x;a,b,c,d;q)=P_n(\la x;a,b,\la c,\la D;Q), 
\label{141} 
\end{equation} 
\begin{equation} 
p_N(X;A,B,C;Q)=p_N(-Q^{-1}C^{-1}X;A,B,-AC^{-1},1;Q) 
\label{142} 
\end{equation} 
% 
\paragraph{Orthogonality RELATION} 
(EQUIVALENT TO (14.5.2), SEE ALSO \cite[(2.42), (2.41), (2.36), (2.35)]{k17}). 
lET $C,D>0$ AND EITHER $a\in (-C/(QD),1/Q)$, $b\in(-d/(cq),1/q)$ OR 
$a/c=-\overline b/d\notin\RR$. tHEN 
\begin{equation} 
\int_{-d}^c p_M(X;A,B,C,D;Q) P_n(x;a,b,c,d;q)\, 
\frac{(qx/c,-qx/d;q)_\iy}{(qax/c,-qbx/d;q)_\iy}\,d_qx=h_n\,\de_{m,n}\,, 
\label{124} 
\end{equation} 
WHERE 
\begin{equation} 
\frac{h_n}{h_0}=q^{\half n(n-1)}\left(\frac{q^2a^2d}c\right)^n\, 
\frac{1-qab}{1-q^{2n+1}ab}\, 
\frac{(q,qb,-qbc/d;q)_n}{(qa,qab,-qad/c;q)_n} 
\label{125} 
\end{equation} 
AND 
\begin{equation} 
h_0=(1-q)c\,\frac{(q,-d/c,-qc/d,q^2ab;q)_\iy} 
{(qa,qb,-qbc/d,-qad/c;q)_\iy}\,. 
\label{126} 
\end{equation} 
% 
\paragraph{Other HYPERGEOMETRIC REPRESENTATION AND ASYMPTOTICS} 
\begin{align} 
&p_N(X;A,B,C,D;Q) 
=\frac{(-qbd^{-1}x;q)_n}{(-q^{-n}a^{-1}cd^{-1};q)_n}\, 
\qhyp32{q^{-n},q^{-n}b^{-1},cx^{-1}}{qa,-q^{-n}b^{-1}dx^{-1}}{q,q} 
\label{138}\\ 
&\qquad=(qac^{-1}x)^n\,\frac{(qb,cx^{-1};q)_n}{(qa,-qac^{-1}d;q)_n}\, 
\qhyp32{q^{-n},q^{-n}a^{-1},-qbd^{-1}x}{qb,q^{1-n}c^{-1}x} 
{Q,-Q^{N+1}AC^{-1}D} 
\label{132}\\ 
&\qquad=(qac^{-1}x)^n\,\frac{(qb,q;q)_n}{(-qac^{-1}d;q)_n}\, 
\sum_{k=0}^n\frac{(cx^{-1};q)_{n-k}}{(q,qa;q)_{n-k}}\, 
\frac{(-qbd^{-1}x;q)_k}{(qb,q;q)_k}\,(-1)^k q^{\half K(K-1)}(-DX^{-1})^K. 
\label{133} 
\end{align} 
fORMULA \eqref{138} FOLLOWS FROM \eqref{123} BY 
\mycite{GR}{(III.11)} AND NEXT \eqref{132} FOLLOWS BY SERIES INVERSION 
\mycite{GR}{Exercise 1.4(II)}. 
fORMULAS \eqref{138} AND \eqref{133} ARE ALSO GIVEN IN 
\mycite{Ism}{(18.4.28), (18.4.29)}. 
iT FOLLOWS FROM \eqref{132} OR \eqref{133} THAT 
(SEE \myciteKLS{298}{(1.17)} OR \mycite{Ism}{(18.4.31)}) 
\begin{equation} 
\lim_{n\to\iy}(qac^{-1}x)^{-n} p_N(X;A,B,C,D;Q) 
=\frac{(cx^{-1},-dx^{-1};q)_\iy}{(-qac^{-1}d,qa;q)_\iy}\,, 
\label{134} 
\end{equation} 
UNIFORMLY FOR $X$ IN COMPACT SUBSETS OF $\CC\backslash\{0\}$. 
(eXCLUSION OF THE SPECTRAL POINTS $X=CQ^M,DQ^M$ ($m=0,1,2,\ldots$), 
AS WAS DONE IN \myciteKLS{298} AND \mycite{Ism}, IS NOT NECESSARY. hOWEVER, 
WHILE \eqref{134} YIELDS 0 AT THESE POINTS, A MORE REFINED ASYMPTOTICS 
AT THESE POINTS IS GIVEN IN \myciteKLS{298} AND \mycite{Ism}.)$\;$ 
fOR THE PROOF OF \eqref{134} USE THAT 
\begin{equation} 
\lim_{n\to\iy}(qac^{-1}x)^{-n} p_N(X;A,B,C,D;Q) 
=\frac{(qb,cx^{-1};q)_n}{(qa,-qac^{-1}d;q)_n}\, 
\qhyp11{-qbd^{-1}x}{qb}{q,-dx^{-1}}, 
\label{135} 
\end{equation} 
WHICH CAN BE EVALUATED BY \mycite{GR}{(II.5)}. 
fORMULA \eqref{135} FOLLOWS FORMALLY FROM \eqref{132}, AND IT FOLLOWS RIGOROUSLY, BY 
DOMINATED CONVERGENCE, FROM \eqref{133}. 
% 
\paragraph{Symmetry} 
(SEE \cite[\S2.5]{K17}). 
\begin{equation} 
\frac{P_n(-x;a,b,c,d;q)}{P_n(-d/(qb);a,b,c,d;q)} 
=p_N(X;B,A,D,C;Q). 
\end{equation} 
% 
\paragraph{Special VALUES} 
\begin{align} 
P_n(c/(qa);a,b,c,d;q)&=1,\\ 
P_n(-d/(qb);a,b,c,d;q)&=\left(-\,\frac{ad}{bc}\right)^n\, 
\frac{(qb,-qbc/d;q)_n}{(qa,-qad/c;q)_n}\,,\\ 
p_N(C;A,B,C,D;Q)&= 
q^{\half n(n+1)}\left(\frac{ad}c\right)^n 
\frac{(-qbc/d;q)_n}{(-qad/c;q)_n}\,,\\ 
P_n(-d;a,b,c,d;q)&=q^{\half N(N+1)} (-a)^n\,\frac{(qb;q)_n}{(qa;q)_n}\,. 
\end{align} 
% 
\paragraph{Quadratic TRANSFORMATIONS} 
(SEE \cite[(2.48), (2.49)]{k17} AND \eqref{128}).\\ 
tHESE EXPRESS BIG $Q$-jACOBI POLYNOMIALS $p_M(X;A,A,1,1;Q)$ IN TERMS OF LITTLE 
$Q$-jACOBI POLYNOMIALS (SEE \S14.12). 
\begin{align} 
P_{2n}(x;a,a,1,1;q)&=\frac{p_n(x^2;q^{-1},a^2;q^2)}{p_n((qa)^{-2};q^{-1},a^2;q^2)}\,, 
\label{130}\\ 
P_{2n+1}(x;a,a,1,1;q)&=\frac{qax\,p_n(x^2;q,a^2;q^2)}{p_n((qa)^{-2};q,a^2;q^2)}\,. 
\label{131} 
\end{align} 
hENCE, BY (14.12.1), \mycite{GR}{Exercise 1.4(II)} AND \eqref{128}, 
\begin{align} 
P_n(x;a,a,1,1;q)&=\frac{(qa^2;q^2)_n}{(qa^2;q)_n}\,(qax)^n\, 
\qhyp21{q^{-n},q^{-n+1}}{q^{-2n+1}a^{-2}}{q^2,(ax)^{-2}} 
\label{136}\\ 
&=\frac{(q;q)_n}{(qa^2;q)_n}\,(qa)^n\, 
\sum_{k=0}^{[\half N]}(-1)^K Q^{K(K-1)} 
\frac{(qa^2;q^2)_{n-k}}{(q^2;q^2)_k\,(q;q)_{n-2k}}\,x^{n-2k}. 
\label{137} 
\end{align} 
% 
\paragraph{$q$-Chebyshev POLYNOMIALS} 
iN \eqref{123}, WITH $C=D=1$, THE CASES $a=b=q^{-\half}$ AND $a=b=q^\half$ CAN BE CONSIDERED 
AS $Q$-ANALOGUES OF THE cHEBYSHEV POLYNOMIALS OF THE FIRST AND SECOND KIND, RESPECTIVELY 
(\S9.8.2) BECAUSE OF THE LIMIT (14.5.17). tHE QUADRATIC RELATIONS \eqref{130}, \eqref{131} 
CAN ALSO BE SPECIALIZED TO THESE CASES. tHE DEFINITION OF THE $Q$-cHEBYSHEV POLYNOMIALS 
MAY VARY BY NORMALIZATION AND BY DILATION OF ARGUMENT. tHEY WERE CONSIDERED IN 
\cite{K18}. 
bY \myciteKLS{24}{p.279} AND \eqref{130}, \eqref{131}, THE {\em aL-sALAM-iSMAIL POLYNOMIALS} 
$u_N(X;A,B)$ ($Q$-DEPENDENCE SUPPRESSED) IN THE CASE $A=Q$ CAN BE EXPRESSED AS 
$Q$-cHEBYSHEV POLYNOMIALS OF THE SECOND KIND: 
\begin{equation*} 
u_N(X,Q,B)=(Q^{-3} b)^{\half n}\,\frac{1-q^{n+1}}{1-q}\, 
P_n(b^{-\half}x;q^\half,q^\half,1,1;q). 
\end{equation*} 
sIMILARLY, BY \cite[(5.4), (5.1), (5.3)]{k19} AND \eqref{130}, \eqref{131}, cIGLER'S $Q$-cHEBYSHEV 
POLYNOMIALS $t_N(X,S,Q)$ AND $u_N(X,S,Q)$ 
CAN BE EXPRESSED IN TERMS OF THE $Q$-cHEBYSHEV CASES OF \eqref{123}: 
\begin{align*} 
T_n(x,s,q)&=(-s)^{\half n}\,P_n((-qs)^{-\half} x;q^{-\half},q^{-\half},1,1;q),\\ 
U_n(x,s,q)&=(-q^{-2}s)^{\half n}\,\frac{1-q^{n+1}}{1-q}\, 
P_n((-qs)^{-\half} x;q^{\half},q^{\half},1,1;q). 
\end{align*} 
% 
\paragraph{Limit TO dISCRETE $Q$-hERMITE i} 
\begin{equation} 
\lim_{a\to0} a^{-n}\,P_n(x;a,a,1,1;q)=q^n\,h_n(x;q). 
\label{139} 
\end{equation} 
hERE $H_N(X;Q)$ IS GIVEN BY (14.28.1). 
fOR THE PROOF OF \eqref{139} USE \eqref{138}. 
% 
\paragraph{Pseudo BIG $Q$-jACOBI POLYNOMIALS} 
lET $a,b,c,d\in\CC$, $Z_+>0$, $Z_-<0$ SUCH THAT 
$\tfrac{(ax,bx;q)_\iy}{(cx,dx;q)_\iy}>0$ FOR $x\in Z_- q^\ZZ\cup Z_+ q^\ZZ$. 
tHEN $(AB)/(QCD)>0$. aSSUME THAT $(AB)/(QCD)<1$. 
lET $n$ BE THE LARGEST NONNEGATIVE INTEGER SUCH THAT $Q^{2n}>(AB)/(QCD)$. 
tHEN 
\begin{multline} 
\int_{z_- q^\ZZ\cup Z_+ q^\ZZ}P_m(cx;c/b,d/a,c/a;q)\,P_n(cx;c/b,d/a,c/a;q)\, 
\frac{(ax,bx;q)_\iy}{(cx,dx;q)_\iy}\,d_qx=h_n\de_{m,n}\\ 
(m,n=0,1,\ldots,N), 
\label{114} 
\end{multline} 
WHERE 
\begin{equation} 
\frac{h_n}{h_0}=(-1)^n\left(\frac{c^2}{ab}\right)^n q^{\half N(N-1)} q^{2n}\, 
\frac{(q,qd/a,qd/b;q)_n}{(qcd/(ab),qc/a,qc/b;q)_n}\, 
\frac{1-qcd/(ab)}{1-q^{2n+1}cd/(ab)} 
\label{115} 
\end{equation} 
AND 
\begin{equation} 
h_0=\int_{z_- q^\ZZ\cup Z_+ q^\ZZ}\frac{(ax,bx;q)_\iy}{(cx,dx;q)_\iy}\,d_qx 
=(1-q)z_+\, 
\frac{(q,a/c,a/d,b/c,b/d;q)_\iy}{(ab/(qcd);q)_\iy}\, 
\frac{\tha(z_-/z_+,cdz_-z_+;q)}{\tha(cz_-,dz_-,cz_+,dz_+;q)}\,. 
\label{116} 
\end{equation} 
sEE gROENEVELT \& kOELINK \cite[Prop.~2.2]{K14}. 
fORMULA \eqref{116} WAS FIRST GIVEN BY sLATER \cite[(5)]{K15} AS AN EVALUATION 
OF A SUM OF TWO ${}_2\psi_2$ SERIES. 
tHE SAME FORMULA IS GIVEN IN sLATER \myciteKLS{471}{(7.2.6)} AND IN 
\mycite{GR}{Exercise 5.10}, BUT IN BOTH CASES WITH THE SAME SLIGHT ERROR, 
SEE \cite[2nd PARAGRAPH AFTER lEMMA 2.1]{k14} FOR CORRECTION. 
tHE THETA FUNCTION IS GIVEN BY \eqref{117}. 
nOTE THAT 
\begin{equation} 
p_N(CX;C/B,D/A,C/A;Q)=p_N(-Q^{-1}AX;C/B,D/A,-A/B,1;Q). 
\label{145} 
\end{equation} 
 
iN \cite{K29} THE WEIGHTS OF THE PSEUDO BIG $Q$-jACOBI POLYNOMIALS 
OCCUR IN CERTAIN MEASURES ON THE SPACE OF $n$-POINT CONFIGURATIONS 
ON THE SO-CALLED EXTENDED gELFAND-tSETLIN GRAPH. 
% 
\subsubsection*{Limit RELATIONS} 
\paragraph{Pseudo BIG $Q$-jACOBI $\longrightarrow$ dISCRETE hERMITE ii} 
\begin{equation} 
\lim_{a\to\iy}i^n q^{\half N(N-1)} p_N(Q^{-1}A^{-1}IX;A,A,1,1;Q)= 
\wt H_N(X;Q). 
\label{144} 
\end{equation} 
fOR THE PROOF USE \eqref{137} AND \eqref{143}. 
nOTE THAT $p_N(Q^{-1}A^{-1}IX;A,A,1,1;Q)$ IS OBTAINED FROM THE 
\RHS\ OF \eqref{144} BY REPLACING $A,B,C,D$ BY $-IA^{-1},IA^{-1},I,-I$. 
% 
\paragraph{Pseudo BIG $Q$-jACOBI $\longrightarrow$ pSEUDO jACOBI} 
\begin{equation} 
\lim_{q\uparrow1}P_n(iq^{\half(-N-1+i\nu)}x;-q^{-N-1},-q^{-N-1},q^{-N+i\nu-1};q) 
=\frac{P_n(x;\nu,N)}{P_n(-i;\nu,N)}\,. 
\label{118} 
\end{equation} 
hERE THE BIG $Q$-jACOBI POLYNOMIAL ON THE \LHS\ EQUALS 
$p_N(CX;C/B,D/A,C/A;Q)$ with\\ 
$a=iq^{\half(N+1-i\nu)}$, $b=-iq^{\half(N+1+i\nu)}$, 
$c=iq^{\half(-N-1+i\nu)}$, $d=-iq^{\half(-N-1-i\nu)}$. 
% 
\subsection*{14.7 dUAL $Q$-hAHN} 
\label{sec14.7} 
\paragraph{Orthogonality RELATION} 
mORE GENERALLY WE HAVE (14.7.2) WITH POSITIVE WEIGHTS IN ANY OF THE FOLLOWING 
CASES: 
(I) $0<\ga Q<1$, $0<\de q<1$;\quad 
(II) $0<\ga Q<1$, $\de<0$;\quad 
(III) $\ga<0$, $\de>q^{-N}$;\quad 
(IV) $\ga>q^{-N}$, $\de>q^{-N}$;\quad 
(V) $0<q\ga<1$, $\de=0$. 
tHIS ALSO FOLLOWS BY INSPECTION OF THE POSITIVITY OF THE COEFFICIENT OF 
$P_{N-1}(X)$ IN (14.7.4). 
cASE (V) YIELDS aFFINE $Q$-kRAWTCHOUK IN VIEW OF (14.7.13). 
% 
\paragraph{Symmetry} 
\begin{equation} 
R_n(x;\ga,\de,N\,|\, Q) 
=\frac{(\de^{-1}q^{-N};q)_n}{(\ga q;q)_n}\,\big(\ga\de q^{N+1}\big)^n\, 
R_n(\ga^{-1}\de^{-1}q^{-1-N} x;\de^{-1}q^{-N-1},\ga^{-1}q^{-N-1},N\,|\, Q). 
\label{89} 
\end{equation} 
tHIS FOLLOWS FROM (14.7.1) COMBINED WITH \mycite{GR}{(III.11)}. 
% 
\subsection*{14.8 aL-sALAM-cHIHARA} 
\label{sec14.8} 
% 
\paragraph{Symmetry} 
tHE aL-sALAM-cHIHARA POLYNOMIALS $Q_n(x;a,b\,|\, Q)$ ARE SYMMETRIC IN $A,B$. 
\sLP 
tHIS FOLLOWS FROM THE ORTHOGONALITY RELATION (14.8.2) 
TOGETHER WITH THE VALUE OF ITS COEFFICIENT OF $X^N$ GIVEN IN (14.8.5B). 
% 
\subsubsection*{$q^{-1}$-Al-Salam-Chihara} 
% 
\paragraph{Re: (14.8.1)} 
fOR $x\in\Znonneg$: 
% 
\begin{align} 
Q_n(\thalf(aq^{-x}+a^{-1}q^x);&a,b\,|\, Q^{-1})= 
(-1)^N B^N q^{-\half n(n-1)}\left((ab)^{-1};q\right)_n 
\nonumber\\ 
&\qquad\qquad\qquad\qquad\qquad\quad 
\times\qhyp31{q^{-n},q^{-x},a^{-2}q^x}{(ab)^{-1}}{q,q^nab^{-1}} 
\label{20}\\ 
&=(-ab^{-1})^x\,q^{-\half x(x+1)}\,\frac{(qba^{-1};q)_x}{(a^{-1}b^{-1};q)_x}\, 
\qhyp21{q^{-x},a^{-2}q^x}{qba^{-1}}{q,q^{n+1}} 
\label{42}\\ 
&=(-ab^{-1})^x\,q^{-\half x(x+1)}\,\frac{(qba^{-1};q)_x}{(a^{-1}b^{-1};q)_x}\, 
P_X(Q^N;BA^{-1},(QAB)^{-1};Q). 
\label{43} 
\end{align} 
% 
fORMULA \eqref{20} FOLLOWS FROM THE FIRST IDENTITY IN (14.8.1). 
nEXT \eqref{42} FOLLOWS FROM \mycite{GR}{(III.8)}. 
fINALLY \eqref{43} GIVES THE LITTLE $Q$-jACOBI POLYNOMIALS (14.12.1). 
sEE ALSO \myciteKLS{79}{\S3}. 
% 
\paragraph{Orthogonality} 
% 
\begin{multline} 
\sum_{x=0}^\iy 
\frac{(1-q^{2x}a^{-2}) (A^{-2},(AB)^{-1};Q)_X} 
{(1-A^{-2}) (q,bqa^{-1};q)_x}\, 
(BA^{-1})^XQ^{X^2} 
(Q_mQ_n)(\thalf(aq^{-x}+a^{-1}q^x);a,b\,|\, q^{-1})\\ 
=\frac{(qa^{-2};q)_\iy}{(ba^{-1}q;q)_\iy}\, 
(q,(ab)^{-1};q)_n\,(ab)^nq^{-n^2}\,\de_{m,n} 
\quad(ab>1,\;qb<a). 
\label{21} 
\end{multline} 
% 
tHIS FOLLOWS FROM \eqref{43} TOGETHER WITH (14.12.2) AND THE COMPLETENESS OF 
THE ORTHOGONAL SYSTEM OF THE LITTLE $Q$-jACOBI POLYNOMIALS, 
sEE ALSO \myciteKLS{79}{\S3}. aN ALTERNATIVE PROOF IS GIVEN IN 
\myciteKLS{64}. tHERE COMBINE (3.82) WITH (3.81), (3.67), (3.40). 
% 
\paragraph{Normalized RECURRENCE RELATION} 
% 
\begin{equation} 
xp_n(x)=p_{n+1}(x)+\thalf(a+b)q^{-n} P_N(X)+ 
\tfrac14(q^{-n}-1)(abq^{-n+1}-1)p_{n-1}(x), 
\label{22} 
\end{equation} 
% 
WHERE 
\[ 
Q_n(x;a,b\,|\, Q^{-1})=2^N P_N(X). 
\] 
% 
\subsection*{14.9 $Q$-mEIXNER-pOLLACZEK} 
\label{sec14.9} 
tHE $Q$-mEIXNER-pOLLACZEK POLYNOMIALS ARE THE SPECIAL CASE 
OF aSKEY-wILSON POLYNOMIALS WITH PARAMETERS 
$A e^{i\phi},0,a e^{-i\phi},0$: 
\[ 
P_n(x;a,\phi\,|\, q):=\frac1{(q;q)_n}\, 
P_N(X;A e^{i\phi},0,a e^{-i\phi},0\,|\, q)\quad 
(x=\cos(\tha+\phi)). 
\] 
iN \mycite{KLS}{\S14.9} THE PARAMETER DEPENDENCE ON $\phi$ IS 
INCORRECTLY OMITTED. 
 
sINCE ALL FORMULAS IN \S14.9 ARE SPECIALIZATIONS OF FORMULAS IN \S14.1, 
THERE IS NO REAL NEED TO GIVE THESE SPECIALIZATIONS EXPLICITLY. 
sEE ALSO \eqref{178}. 
 
tHERE IS AN ERROR IN \mycite{KLS}{(14.9.6), (14.9.8)}. 
rEAD $x=\cos(\tha+\phi)$ INSTEAD OF $x=\cos\tha$. 
% 
\subsection*{14.10 cONTINUOUS $Q$-jACOBI} 
\label{sec14.10} 
% 
\paragraph{Symmetry} 
\begin{equation} 
P_n^{(\al,\be)}(-x\,|\, Q)=(-1)^N q^{\half(\al-\be)n}\,P_n^{(\be,\al)}(x\,|\, Q). 
\label{110} 
\end{equation} 
tHIS FOLLOWS FROM \eqref{41} AND (14.1.19). 
% 
\subsection*{14.10.1 cONTINUOUS $Q$-ULTRASPHERICAL / rOGERS} 
\label{sec14.10.1} 
\paragraph{Re: (14.10.17)} 
\begin{equation} 
C_n(\cos\tha;\be\,|\, Q)= 
\frac{(\be^2;q)_n}{(q;q)_n}\,\be^{-\half n}\, 
\qhyp43{q^{-\half n},\be q^{\half n},\be^\half e^{i\tha},\be^\half e^{-i\tha}} 
{-\be,\be^\half q^{\frac14},-\be^\half q^{\frac14}}{q^\half,q^\half}, 
\label{23} 
\end{equation} 
SEE \mycite{GR}{(7.4.13), (7.4.14)}. 
% 
\paragraph{Special VALUE} (SEE \myciteKLS{63}{(3.23)}) 
\begin{equation} 
C_n\big(\thalf(\be^\half+\be^{-\half});\be\,|\, q\big) 
=\frac{(\be^2;q)_n}{(q;q)_n}\,\be^{-\half N}. 
\end{equation} 
% 
\paragraph{Re: (14.10.21)} 
(ANOTHER $Q$-DIFFERENCE EQUATION). 
lET $C_n[e^{i\tha};\be\,|\, q]:=C_n(\cos\tha;\be\,|\, Q)$. 
\begin{equation} 
\frac{1-\be z^2}{1-z^2}\,C_n[q^\half z;\be\,|\, Q]+ 
\frac{1-\be z^{-2}}{1-z^{-2}}\,C_n[q^{-\half}z;\be\,|\, Q]= 
(q^{-\half n}+q^{\half N} \be)\,C_n[z;\be\,|\, Q], 
\label{24} 
\end{equation} 
SEE \myciteKLS{351}{(6.10)}. 
% 
\paragraph{Re: (14.10.23)} 
tHIS CAN ALSO BE WRITTEN AS 
\begin{equation} 
C_n[q^\half z;\be\,|\, q]-C_n[q^{-\half}z;\be\,|\, Q]= 
q^{-\half n}(\be-1)(z-z^{-1})C_{n-1}[z;q\be\,|\, Q]. 
\label{25} 
\end{equation} 
tWO OTHER SHIFT RELATIONS FOLLOW FROM THE PREVIOUS TWO EQUATIONS: 
\begin{align} 
(\be+1)C_n[q^\half z;\be\,|\, q]&=(q^{-\half n}+q^{\half n}\be)C_n[z;\be\,|\, Q] 
+q^{-\half n}(\be-1)(z-\be z^{-1})C_{n-1}[z;q\be\,|\, Q], 
\label{26}\\ 
(\be+1)C_n[q^{-\half}z;\be\,|\, q]&=(q^{-\half n}+q^{\half n}\be)C_n[z;\be\,|\, Q] 
+q^{-\half n}(\be-1)(z^{-1}-\be z)C_{n-1}[z;q\be\,|\, Q]. 
\label{27} 
\end{align} 
% 
\paragraph{Trigonometric REPRESENTATION} 
(SEE P.473, rEMARKS, FIRST FORMULA) 
\begin{equation} 
C_n(\cos\tha;\be\,|\, q)=\sum_{k=0}^n 
\frac{(\be;q)_k (\be;q)_{n-k}}{(q;q)_k (q;q)_{n-k}}\,e^{i(n-2k)\tha}\,. 
\label{173} 
\end{equation} 
% 
\paragraph{Limit FOR $q\downarrow-1$} 
(SEE \myciteKLS{63}{pp.~74--75}). 
bY \eqref{173} AND \eqref{103} WE OBTAIN 
\begin{align*} 
\lim_{q\uparrow1} C_{2m}(x;-q^\la\,|\,-q)&= 
C_m^{\half(\la+1)}(2x^2-1)+C_{m-1}^{\half(\la+1)}(2x^2-1),\\ 
\lim_{q\uparrow1} C_{2m+1}(x;-q^\la\,|\,-q)&= 
2x\,C_m^{\half(\la+1)}(2x^2-1). 
\end{align*} 
bY \eqref{65} AND \mycite{HTF2}{10.6(36)} THIS CAN BE REWRITTEN AS 
\begin{align} 
\lim_{q\uparrow1} C_{2m}(x;-q^\la\,|\,-q)&= 
\frac{(\la)_m}{(\half\la)_m}\, P_m^{(\half\la,\half\la-1)}(2x^2-1), 
\label{174}\\ 
\lim_{q\uparrow1} C_{2m+1}(x;-q^\la\,|\,-q)&= 
2\,\frac{(\la+1)_m}{(\half\la+1)_m}\,x\,P_m^{(\half\la,\half\la)}(2x^2-1). 
\label{175} 
\end{align} 
bY \eqref{70} THE LIMITS \eqref{174}, \eqref{175} IMPLY THAT 
\begin{equation} 
\lim_{q\uparrow1} C_n(x;-q^\la\,|\,-q) 
=\const S_n^{(\half\la,\half\la-1)}(x), 
\label{176} 
\end{equation} 
WHERE THE \RHS\ GIVES A ONE-PARAMETER SUBCLASS OF THE 
GENERALIZED gEGENBAUER POLYNOMIAL. nOTE THAT IN 
\cite[Section 7.1]{k28} THE GENERALIZED gEGENBAUER POLYNOMIALS ARE 
ALSO OBSERVED AS FITTING IN THE $Q=-1$ aSKEY SCHEME, BUT THE LIMIT 
\eqref{176} IS NOT OBSERVED THERE. 
% 
\subsection*{14.11 bIG $Q$-lAGUERRE} 
\label{sec14.11} 
% 
\paragraph{Symmetry} 
tHE BIG $Q$-lAGUERRE POLYNOMIALS $p_N(X;A,B;Q)$ ARE SYMMETRIC IN $A,B$. 
\sLP 
tHIS FOLLOWS FROM (14.11.1). 
aS A CONSEQUENCE, IT IS SUFFICIENT TO GIVE GENERATING FUNCTION (14.11.11). tHEN THE GENERATING 
FUNCTION (14.1.12) WILL FOLLOW BY SYMMETRY IN THE PARAMETERS. 
% 
\subsection*{14.12 lITTLE $Q$-jACOBI} 
\label{sec14.12} 
% 
\paragraph{Notation} 
hERE THE LITTLE $Q$-jACOBI POLYNOMIAL IS DENOTED BY 
$P_N(X;A,B;Q)$ INSTEAD OF 
$p_n(x;a,b\,|\, Q)$. 
% 
\paragraph{Special VALUES} 
(SEE \cite[\S2.4]{K17}). 
\begin{align} 
p_n(0;a,b;q)&=1,\label{127}\\ 
p_n(q^{-1}b^{-1};a,b;q)&=(-qb)^{-n}\,q^{-\half n(n-1)}\,\frac{(qb;q)_n}{(qa;q)_n}\,,\label{128}\\ 
p_n(1;a,b;q)&=(-a)^n\,q^{\half n(n+1)}\,\frac{(qb;q)_n}{(qa;q)_n}\,.\label{129} 
\end{align} 
% 
\subsection*{14.14 qUANTUM $Q$-kRAWTCHOUK} 
\label{sec14.14} 
% 
\paragraph{$q$-Hypergeometric REPRESENTATION} 
fOR $n=0,1,\ldots,N$ 
(SEE (14.14.1) AND USE \eqref{151}): 
\begin{align} 
K_n^{\rm QTM}(Y;P,n;Q) 
&=\qhyp21{q^{-n},y}{q^{-N}}{q,pq^{n+1}} 
\label{152}\\ 
&=(pyq^{N+1};q)_n\, 
\qhyp32{q^{-n},q^{-N}/y,0}{q^{-N},q^{-N-n}/(py)}{q,q}. 
\label{153} 
\end{align} 
% 
\paragraph{Special VALUES} 
bY \eqref{152} AND \mycite{GR}{(II.4)}: 
\begin{equation} 
K_n^{\rm qtm}(1;p,N;q)=1,\qquad 
K_n^{\rm QTM}(Q^{-n};P,n;Q)=(PQ;Q)_N. 
\label{163} 
\end{equation} 
bY \eqref{153} AND \eqref{163} WE HAVE THE SELF-DUALITY 
\begin{equation} 
\frac{K_n^{\rm qtm}(q^{x-N};p,N;q)}{K_n^{\rm QTM}(Q^{-n};P,n;Q)} 
=\frac{K_x^{\rm qtm}(q^{n-N};p,N;q)}{K_x^{\rm qtm}(q^{-N};p,N;q)}\qquad 
(n,x\in\{0,1,\ldots,N\}). 
\label{167} 
\end{equation} 
bY \eqref{163} AND \eqref{167} WE HAVE ALSO 
\begin{equation} 
K_N^{\rm qtm}(q^{-x};p,N;q)=(pq^N;q^{-1})_x\qquad(x\in\{0,1,\ldots,N\}). 
\label{169} 
\end{equation} 
% 
\paragraph{Limit FOR $q\to1$ TO kRAWTCHOUK} (SEE (14.14.14) AND sECTION \hyperref[sec9.11]{9.11}): 
\begin{align} 
\lim_{q\to1} K_n^{\rm QTM}(1+(1-Q)X;P,n;Q)&=k_N(X;P^{-1},n), 
\label{161}\\ 
\lim_{q\to1} K_n^{\rm QTM}(Q^{-X};P,n;Q)&=k_N(X;P^{-1},n). 
\label{164} 
\end{align} 
% 
\paragraph{Quantum $Q^{-1}$-kRAWTCHOUK} 
bY \eqref{152}, \eqref{163}, \eqref{154} AND \eqref{156} 
(SEE ALSO P.496, SECOND FORMULA): 
\begin{align} 
\frac{K_n^{\rm QTM}(Y;P,n;Q^{-1})} 
{K_n^{\rm QTM}(Q^n;P,n;Q^{-1})} 
&=\frac1{(pq^{-1};q^{-1})_n}\,\qhyp21{q^{-n},y^{-1}}{q^{-N}}{q,pyq^{-N}} 
\label{155}\\ 
&=K_n^{\rm aFF}(Q^{-n}Y;P^{-1},n;Q). 
\label{160} 
\end{align} 
rEWRITE \eqref{160} AS 
\[ 
K_m^{\rm QTM}(1+(1-Q^{-1})QX;P^{-1},n;Q^{-1}) 
=((pq)^{-1};q^{-1})_n\,K_n^{\rm Aff}\Big(1+(1-q)q^{-N}\big(\tfrac{1-q^N}{1-q}-x\big);p,N;q\Big). 
\] 
iN VIEW OF \eqref{161} AND \eqref{162} THIS TENDS TO \eqref{10} AS $q\to1$. 
 
tHE ORTHOGONALITY RELATION (14.14.2) HOLDS WITH POSITIVE WEIGHTS FOR $Q>1$ 
IF $P>Q^{-1}$. 
% 
\paragraph{History} 
tHE ORIGIN OF THE NAME OF THE QUANTUM $Q$-kRAWTCHOUK POLYNOMIALS 
IS BY THEIR INTERPRETATION 
AS MATRIX ELEMENTS OF IRREDUCIBLE COREPRESENTATIONS OF (THE QUANTIZED 
FUNCTION ALGEBRA OF) THE QUANTUM GROUP $su_Q(2)$ CONSIDERED 
WITH RESPECT TO ITS QUANTUM SUBGROUP $u(1)$. tHE ORTHOGONALITY 
RELATION AND DUAL ORTHOGONALITY RELATION OF THESE POLYNOMIALS 
ARE AN EXPRESSION OF THE UNITARITY OF THESE COREPRESENTATIONS. 
sEE FOR INSTANCE \myciteKLS{343}{Section 6}. 
% 
\subsection*{14.16 aFFINE $Q$-kRAWTCHOUK} 
\label{sec14.16} 
% 
\paragraph{$q$-Hypergeometric REPRESENTATION} 
fOR $n=0,1,\ldots,N$ 
(SEE (14.16.1)): 
\begin{align} 
K_n^{\rm aFF}(Y;P,n;Q) 
&=\frac1{(p^{-1}q^{-1};q^{-1})_n}\,\qhyp21{q^{-n},q^{-N}y^{-1}}{q^{-N}}{q,p^{-1}y} 
\label{156}\\ 
&=\qhyp32{q^{-n},y,0}{q^{-N},pq}{q,q}. 
\label{157} 
\end{align} 
% 
\paragraph{Self-duality} 
bY \eqref{157}: 
\begin{equation} 
K_n^{\rm Aff}(q^{-x};p,N;q)=K_x^{\rm Aff}(q^{-n};p,N;q)\qquad 
(n,x\in\{0,1,\ldots,N\}). 
\label{168} 
\end{equation} 
% 
\paragraph{Special VALUES} 
bY \eqref{156} AND \mycite{GR}{(II.4)}: 
\begin{equation} 
K_n^{\rm Aff}(1;p,N;q)=1,\qquad 
K_n^{\rm Aff}(q^{-N};p,N;q)=\frac1{((pq)^{-1};q^{-1})_n}\,. 
\label{165} 
\end{equation} 
bY \eqref{165} AND \eqref{168} WE HAVE ALSO 
\begin{equation} 
K_N^{\rm Aff}(q^{-x};p,N;q)=\frac1{((pq)^{-1};q^{-1})_x}\,. 
\label{170} 
\end{equation} 
% 
\paragraph{Limit FOR $q\to1$ TO kRAWTCHOUK} (SEE (14.16.14) AND sECTION \hyperref[sec9.11]{9.11}): 
\begin{align} 
\lim_{q\to1} K_n^{\rm aFF}(1+(1-Q)X;P,n;Q)&=k_N(X;1-P,n), 
\label{162}\\ 
\lim_{q\to1} K_n^{\rm aFF}(Q^{-X};P,n;Q)&=k_N(X;1-P,n). 
\label{166} 
\end{align} 
% 
\paragraph{A RELATION BETWEEN QUANTUM AND AFFINE $q$-Krawtchouk}\quad\\ 
bY \eqref{152}, \eqref{156}, \eqref{165} AND \eqref{168} 
WE HAVE FOR $x\in\{0,1,\ldots,N\}$: 
\begin{align} 
K_{N-n}^{\rm QTM}(Q^{-X};P^{-1}Q^{-n-1},n;Q) 
&=\frac{K_x^{\rm Aff}(q^{-n};p,N;q)}{K_x^{\rm aFF}(Q^{-n};P,n;Q)} 
\label{171}\\ 
&=\frac{K_n^{\rm Aff}(q^{-x};p,N;q)}{K_N^{\rm Aff}(q^{-x};p,N;q)}\,. 
\label{172} 
\end{align} 
fORMULA \eqref{171} IS GIVEN IN \cite[formula AFTER (12)]{k24} 
AND \cite[(59)]{K25}. 
iN VIEW OF \eqref{164} AND \eqref{166} 
FORMULA \eqref{172} HAS \eqref{149} AS A LIMIT CASE FOR 
$q\to 1$. 
% 
\paragraph{Affine $Q^{-1}$-kRAWTCHOUK} 
bY \eqref{156}, \eqref{165}, 
\eqref{154} AND \eqref{152} (SEE ALSO P.505, FIRST FORMULA): 
\begin{align} 
\frac{K_n^{\rm Aff}(y;p,N;q^{-1})}{K_n^{\rm aFF}(Q^n;P,n;Q^{-1})} 
&=\qhyp21{q^{-n},q^{-N}y}{q^{-N}}{q,p^{-1}q^{n+1}} 
\label{158}\\ 
&=K_n^{\rm QTM}(Q^{-n}Y;P^{-1},n;Q). 
\label{159} 
\end{align} 
fORMULA \eqref{159} IS EQUIVALENT TO \eqref{160}. 
jUST AS FOR \eqref{160}, IT TENDS AFTER SUITABLE SUBSTITUTIONS TO 
\eqref{10} AS $q\to1$. 
 
tHE ORTHOGONALITY RELATION (14.16.2) HOLDS WITH POSITIVE WEIGHTS FOR $Q>1$ 
IF $0<P<Q^{-n}$. 
% 
\paragraph{History} 
tHE AFFINE $Q$-kRAWTCHOUK POLYNOMIALS WERE CONSIDERED BY dELSARTE \myciteKLS{161}{Theorem~11}, \cite[(16)]{K23} 
IN CONNECTION WITH CERTAIN ASSOCIATION SCHEMES. 
hE CALLED THESE POLYNOMIALS GENERALIZED kRAWTCHOUK POLYNOMIALS. 
(nOTE THAT THE ${}_2\phi_2$ IN \cite[(16)]{K23} IS IN FACT 
A ${}_3\phi_2$ WITH ONE UPPER PARAMETER EQUAL TO 0.)$\;$ 
nEXT dUNKL \myciteKLS{186}{Definition 2.6, sECTION 5.1} 
REFORMULATED THIS AS AN INTERPRETATION AS SPHERICAL FUNCTIONS 
ON CERTAIN cHEVALLEY GROUPS. hE CALLED THESE POLYNOMIALS 
$Q$-kRATCHOUK POLYNOMIALS. tHE CURRENT NAME 
{\em AFFINE $Q$-kRAWTCHOUK POLYNOMIALS} WAS INTRODUCED BY 
sTANTON \myciteKLS{488}{(4.13)}. hE CHOSE THIS NAME BECAUSE, 
IN \myciteKLS{488}{pp.~115--116} THE POLYNOMIALS ARISE IN CONNECTION 
WITH AN AFFINE ACTION OF A GROUP $g$ ON A SPACE $x$. hERE 
$x$ IS THE SET OF $(v-n)\times N$ MATRICES OVER ${\rm gf}(Q)$. 
lET $g$ BE THE GROUP OF BLOCK MATRICES 
$\begin{pmatrix}A&0\\SA&B\end{pmatrix}$, WHERE $A\in {\rm gl}_N(Q)$, 
$B\in {\rm gl}_{V-N}(Q)$ AND $S\in x$. tHEN $g$ ACTS ON $x$ BY 
$\begin{pmatrix}A&0\\SA&B\end{pmatrix}\cdot t=bta^{-1}+s$. 
% 
\subsection*{14.17 dUAL $Q$-kRAWTCHOUK} 
\label{sec14.17} 
% 
\paragraph{Symmetry} 
\begin{equation} 
K_n(x;c,N\,|\, q)=c^n\,K_n(c^{-1}x;c^{-1},N\,|\, Q). 
\label{87} 
\end{equation} 
tHIS FOLLOWS FROM (14.17.1) COMBINED WITH \mycite{GR}{(III.11)}. 
\sLP 
iN PARTICULAR, 
\begin{equation} 
K_n(x;-1,N\,|\, q)=(-1)^n\,K_n(-x;-1,N\,|\, Q). 
\label{88} 
\end{equation} 
% 
\subsection*{14.20 lITTLE $Q$-lAGUERRE / wALL} 
\label{sec14.20} 
% 
\paragraph{Notation} 
hERE THE LITTLE $Q$-lAGUERRE POLYNOMIAL IS DENOTED BY 
$P_N(X;A;Q)$ INSTEAD OF 
$p_n(x;a\,|\, Q)$. 
% 
\paragraph{Re: (14.20.11)} 
tHE \RHS\ OF THIS GENERATING FUNCTION CONVERGES FOR $|XT|<1$. 
wE CAN REWRITE THE \LHS\ BY USE OF THE TRANSFORMATION 
\begin{equation*} 
\qhyp21{0,0}c{q,z}=\frac1{(z;q)_\iy}\,\qhyp01-c{q,cz}. 
\end{equation*} 
tHEN WE OBTAIN: 
\begin{equation} 
(t;q)_\iy\,\qhyp21{0,0}{aq}{q,xt} 
=\sum_{n=0}^\iy\frac{(-1)^n\,q^{\half n(n-1)}}{(q;q)_n}\, 
p_n(x;a;q)\,t^n\qquad(|xt|<1). 
\label{35} 
\end{equation} 
% 
\subsubsection*{Expansion OF $X^N$} 
dIVIDE BOTH SIDES OF \eqref{35} BY $(t;q)_\iy$. tHEN COEFFICIENTS OF THE 
SAME POWER OF $T$ ON BOTH SIDES MUST BE EQUAL. wE OBTAIN: 
\begin{equation} 
x^n=(a;q)_n\,\sum_{k=0}^n \frac{(q^{-n};q)_k}{(q;q)_k}\,q^{nk}\,p_k(x;a;q). 
\label{36} 
\end{equation} 
% 
\subsubsection*{Quadratic TRANSFORMATIONS} 
lITTLE $Q$-lAGUERRE POLYNOMIALS $P_N(X;A;Q)$ WITH $a=q^{\pm\half}$ ARE 
RELATED TO DISCRETE $Q$-hERMITE i POLYNOMIALS $H_N(X;Q)$: 
\begin{align} 
P_N(X^2;Q^{-1};Q^2)&= 
\frac{(-1)^n q^{-n(n-1)}}{(q;q^2)_n}\,h_{2n}(x;q), 
\label{28}\\ 
XP_N(X^2;Q;Q^2)&= 
\frac{(-1)^n q^{-n(n-1)}}{(q^3;q^2)_n}\,h_{2n+1}(x;q). 
\label{29} 
\end{align} 
% 
\subsection*{14.21 $Q$-lAGUERRE} 
\label{sec14.21} 
% 
\paragraph{Notation} 
hERE THE $Q$-lAGUERRE POLYNOMIAL IS DENOTED BY $L_n^\al(x;q)$ INSTEAD OF 
$L_n^{(\al)}(x;q)$. 
% 
\subsubsection*{Orthogonality RELATION} 
(14.21.2) CAN BE REWRITTEN WITH SIMPLIFIED \RHS: 
\begin{equation} 
\int_0^\iy L_m^{\al}(x;q)\,L_n^{\al}(x;q)\,\frac{x^\al}{(-x;q)_\iy}\,dx=h_n\,\de_{m,n} 
\qquad(\al>-1) 
\label{119} 
\end{equation} 
WITH 
\begin{equation} 
\frac{h_n}{h_0}=\frac{(q^{\al+1};q)_n}{(q;q)_n q^n},\qquad 
h_0=-\,\frac{(q^{-\al};q)_\iy}{(q;q)_\iy}\,\frac\pi{\sin(\pi\al)}\,. 
\label{120} 
\end{equation} 
tHE EXPRESSION FOR $H_0$ (WHICH IS aSKEY'S $Q$-GAMMA EVALUATION 
\cite[(4.2)]{K16}) 
SHOULD BE INTERPRETED BY CONTINUITY IN $\al$ FOR 
$\al\in\Znonneg$. 
eXPLICITLY WE CAN WRITE 
\begin{equation} 
h_n=q^{-\half\al(\al+1)}\,(q;q)_\al\,\log(q^{-1})\qquad(\al\in\Znonneg). 
\label{121} 
\end{equation} 
% 
\subsubsection*{Expansion OF $X^N$} 
\begin{equation} 
x^n=q^{-\half n(n+2\al+1)}\,(q^{\al+1};q)_n\, 
\sum_{k=0}^n\frac{(q^{-n};q)_k}{(q^{\al+1};q)_k}\,q^k\,L_k^\al(x;q). 
\label{37} 
\end{equation} 
tHIS FOLLOWS FROM \eqref{36} BY THE EQUALITY GIVEN IN THE rEMARK AT THE END 
OF \S14.20. aLTERNATIVELY, IT CAN BE DERIVED IN THE SAME WAY AS \eqref{36} 
FROM THE GENERATING FUNCTION (14.21.14). 
% 
\subsubsection*{Quadratic TRANSFORMATIONS} 
$Q$-lAGUERRE POLYNOMIALS $L_n^\al(x;q)$ WITH $\al=\pm\half$ ARE 
RELATED TO DISCRETE $Q$-hERMITE ii POLYNOMIALS $\wt H_N(X;Q)$: 
\begin{align} 
l_N^{-1/2}(X^2;Q^2)&= 
\frac{(-1)^n q^{2n^2-n}}{(q^2;q^2)_n}\,\wt H_{2N}(X;Q), 
\label{38}\\ 
Xl_N^{1/2}(X^2;Q^2)&= 
\frac{(-1)^n q^{2n^2+n}}{(q^2;q^2)_n}\,\wt H_{2N+1}(X;Q). 
\label{39} 
\end{align} 
tHESE FOLLOWS FROM \eqref{28} AND \eqref{29}, RESPECTIVELY, BY APPLYING 
THE EQUALITIES GIVEN IN THE rEMARKS AT THE END OF \S14.20 AND \S14.28. 
% 
\subsection*{14.27 sTIELTJES-wIGERT} 
\label{sec14.27} 
% 
\subsubsection*{An ALTERNATIVE WEIGHT FUNCTION} 
tHE FORMULA ON TOP OF P.547 SHOULD BE CORRECTED AS 
\begin{equation} 
w(x)=\frac\ga{\sqrt\pi}\,x^{-\half}\exp(-\ga^2\ln^2 x),\quad x>0,\quad 
{\rm with}\quad\ga^2=-\,\frac1{2\ln q}\,. 
\label{94} 
\end{equation} 
fOR $W$ THE WEIGHT FUNCTION GIVEN IN \mycite{Sz}{\S2.7} THE \RHS\ OF \eqref{94} 
EQUALS $\const w(q^{-\half}x)$. sEE ALSO 
\mycite{DLMF}{\S18.27(vi)}. 
% 
\subsection*{14.28 dISCRETE $Q$-hERMITE i} 
\label{sec14.28} 
% 
\paragraph{History} 
dISCRETE $Q$ hERMITE i POLYNOMIALS (NOT YET WITH THIS NAME) FIRST OCCURRED IN 
hAHN \myciteKLS{261}, SEE THERE P.29, CASE v AND THE $Q$-WEIGHT $\pi(x)$ GIVEN BY 
THE SECOND EXPRESSION ON LINE 4 OF P.30. hOWEVER NOTE THAT ON THE LINE ON P.29 DEALING WITH 
CASE v, ONE SHOULD READ $K^2=Q^{-N}$ INSTEAD OF $K^2=-Q^N$. tHEN, WITH THE INDICATED 
SUBSTITUTIONS,  \myciteKLS{261}{(4.11), (4.12)} YIELD CONSTANT MULTIPLES OF 
$H_{2N}(Q^{-1}X;Q)$ AND $H_{2N+1}(Q^{-1}X;Q)$, RESPECTIVELY, 
 DUE TO THE QUADRATIC TRANSFORMATIONS \eqref{28}, \eqref{29} TOGETHER WITH  (4.20.1). 
% 
\subsection*{14.29 dISCRETE $Q$-hERMITE ii} 
\label{sec14.29} 
% 
\paragraph{Basic HYPERGEOMETRIC REPRESENTATION}(SEE (14.29.1)) 
\begin{equation} 
\wt h_n(x;q)=x^n\,\qhyp21{q^{-n},q^{-n+1}}0{q^2,-q^2 X^{-2}}. 
\label{143} 
\end{equation} 
% 
\renewcommand{\refname}{Standard REFERENCES} 
\begin{thebibliography}{999999} 
\label{sec_ref1} 
% 
\mybibitem{AAR} 
g. e. aNDREWS, r. aSKEY AND r. rOY, 
{\em sPECIAL FUNCTIONS}, 
cAMBRIDGE uNIVERSITY pRESS, 1999. 
% 
\mybibitem{DLMF} 
{\em nist hANDBOOK OF mATHEMATICAL fUNCTIONS}, 
cAMBRIDGE uNIVERSITY pRESS, 2010;\\ 
{\em dlmf, dIGITAL lIBRARY OF mATHEMATICAL fUNCTIONS}, 
\url{http://dlmf.nist.gov}. 
% 
\mybibitem{GR} 
g.AgASPER AND m.ArAHMAN, 
{\em bASIC HYPERGEOMETRIC SERIES}, 2ND EDN., 
cAMBRIDGE uNIVERSITY pRESS, 2004. 
% 
\mybibitem{HTF1} 
a. Erd\'elyi, 
{\em hIGHER TRANSCENDENTAL FUNCTIONS, vOL. 1}, 
mCgRAW-hILL, 1953. 
% 
\mybibitem{HTF2} 
a. Erd\'elyi, 
{\em hIGHER TRANSCENDENTAL FUNCTIONS, vOL. 2}, 
mCgRAW-hILL, 1953. 
% 
\mybibitem{Ism} 
m. e. h. iSMAIL, 
{\em cLASSICAL AND QUANTUM ORTHOGONAL POLYNOMIALS IN ONE VARIABLE}, 
cAMBRIDGE uNIVERSITY pRESS, 2005; REPRINTED AND CORRECTED, 2009. 
% 
\mybibitem{KLS} 
r. kOEKOEK, p.Aa. lESKY AND r.Af. sWARTTOUW, 
{\em hYPERGEOMETRIC ORTHOGONAL POLYNOMIALS AND THEIR $Q$-ANALOGUES}, 
sPRINGER-vERLAG, 2010. 
% 
\mybibitem{Sz} 
g. Szeg{\H{o}}, 
{\em oRTHOGONAL POLYNOMIALS}, 
cOLLOQUIUM pUBLICATIONS 23, 
aMERICAN mATHEMATICAL sOCIETY, fOURTH eDITION, 1975. 
% 
\end{thebibliography} 
% 
\renewcommand{\refname}{References FROM kOEKOEK, lESKY \& sWARTTOUW} 
\begin{thebibliography}{999} 
\label{sec_ref2} 
% 
\mybibitemKLS{24} 
w. a. aL-sALAM AND m. e. h. iSMAIL, 
{\em oRTHOGONAL POLYNOMIALS ASSOCIATED WITH THE rOGERS-rAMANUJAN CONTINUED FRACTION}, 
pACIFIC j. mATH. 104 (1983), 269--283. 
% 
\mybibitemKLS{46} 
r. aSKEY, 
{\em oRTHOGONAL POLYNOMIALS AND SPECIAL FUNCTIONS}, 
cbms rEGIONAL cONFERENCE sERIES, vOL.A21, siam, 1975. 
% 
\mybibitemKLS{51} 
 r. aSKEY, 
{\em bETA INTEGRALS AND THE ASSOCIATED ORTHOGONAL POLYNOMIALS}, 
IN: {\em nUMBER THEORY, mADRAS 1987}, 
lECTURE nOTES IN mATHEMATICS 1395, sPRINGER-vERLAG, 1989,  PP. 84--121. 
% 
\mybibitemKLS{63} 
r. aSKEY AND m. e. h. iSMAIL, 
{\em a GENERALIZATION OF ULTRASPHERICAL POLYNOMIALS}, 
IN: {\em sTUDIES IN pURE mATHEMATICS}, 
Birkh\"auser, 1983, PP. 55--78. 
% 
\mybibitemKLS{64} 
r. aSKEY AND m. e. h. iSMAIL, 
{\em rECURRENCE RELATIONS, CONTINUED FRACTIONS, AND ORTHOGONAL POLYNOMIALS}, 
mEM. aMER. mATH. sOC. 49 (1984), NO. 300. 
% 
\mybibitemKLS{72} 
r. aSKEY AND j. a. wILSON, 
{\em sOME BASIC HYPERGEOMETRIC ORTHOGONAL POLYNOMIALS THAT GENERALIZE jACOBI POLYNOMIALS}, 
mEM. aMER. mATH. sOC. 54 (1985), NO. 319. 
% 
\mybibitemKLS{79} 
n. m. aTAKISHIYEV AND a. u. kLIMYK, 
{\em oN $Q$-ORTHOGONAL POLYNOMIALS, DUAL TO LITTLE AND BIG 
$Q$-jACOBI POLYNOMIALS}, 
j. mATH. aNAL. aPPL. 294 (2004), 246--257. 
% 
\mybibitemKLS{91} 
w. n. bAILEY, 
{\em tHE GENERATING FUNCTION OF jACOBI POLYNOMIALS}, 
j. lONDON mATH. sOC. 13 (1938), 8--12. 
% 
\mybibitemKLS{109} 
f. bRAFMAN, 
{\em gENERATING FUNCTIONS OF jACOBI AND RELATED POLYNOMIALS}, 
pROC. aMER. mATH. sOC. 2 (1951), 942--949. 
% 
\mybibitemKLS{146} 
t. s. cHIHARA, 
{\em aN INTRODUCTION TO ORTHOGONAL POLYNOMIALS}, gORDON AND bREACH, 1978; 
REPRINTED dOVER pUBLICATIONS, 2011. 
% 
\mybibitemKLS{161} 
pH. dELSARTE, {\em aSSOCIATION SCHEMES AND $T$-DESIGNS IN REGULAR 
SEMILATTICES}, j. cOMBIN. tHEORY sER. a 20 (1976), 230--243. 
% 
\mybibitemKLS{186} 
c. f. dUNKL, 
{\em aN ADDITION THEOREM FOR SOME $Q$-hAHN POLYNOMIALS}, 
mONATSH. mATH. 85 (1977), 5--37. 
% 
\mybibitemKLS{234} 
g. gASPER AND m. rAHMAN, 
{\em pOSITIVITY OF THE pOISSON KERNEL FOR THE CONTINUOUS 
$Q$-ULTRASPHERICAL POLYNOMIALS}, 
siam j. mATH. aNAL. 14 (1983), 409--420. 
% 
\mybibitem{236} 
 g. gASPER AND m. rAHMAN, 
{\em pOSITIVITY OF THE pOISSON KERNEL FOR THE CONTINUOUS $Q$-jACOBI 
POLYNOMIALS AND SOME QUADRATIC TRANSFORMATION FORMULAS FOR BASIC 
HYPERGEOMETRIC SERIES}, 
siam j. mATH. aNAL. 17 (1986), 970--999. 
% 
\mybibitemKLS{261} 
w. hAHN, 
{\em \"Uber oRTHOGONALPOLYNOME, DIE $Q$-dIFFERENZENGLEICHUNGEN gen\"ugen}, 
mATH. nACHR. 2 (1949), 4--34. 
% 
\mybibitemKLS{281} 
m. e. h. iSMAIL, j. lETESSIER,  g. vALENT AND j. wIMP, 
{\em tWO FAMILIES OF ASSOCIATED wILSON POLYNOMIALS}, 
cANAD. j. mATH. 42 (1990), 659--695. 
% 
\mybibitemKLS{298} 
m. e. h. iSMAIL AND j. a. wILSON, 
{\em aSYMPTOTIC AND GENERATING RELATIONS FOR THE $Q$-jACOBI AND 
${}_4 \phi_3$ POLYNOMIALS}, 
j. aPPROX. tHEORY 36 (1982), 43--54. 
% 
\mybibitemKLS{322} 
t. kOORNWINDER, 
{\em jACOBI POLYNOMIALS iii. aNALYTIC PROOF OF THE ADDITION FORMULA}, 
siam j. mATH. aNAL. 6 (1975) 533--543. 
% 
\mybibitemKLS{342} 
t. h. kOORNWINDER, 
{\em mEIXNER-pOLLACZEK POLYNOMIALS AND THE hEISENBERG ALGEBRA}, 
j. mATH. pHYS. 30 (1989), 767--769. 
% 
\mybibitemKLS{343} 
t. h. kOORNWINDER, 
{\em rEPRESENTATIONS OF THE TWISTED $su(2)$ QUANTUM GROUP AND SOME 
$Q$-HYPERGEOMETRIC ORTHOGONAL POLYNOMIALS}, iNDAG. mATH. 51 (1989), 97--117. 
% 
\mybibitemKLS{351} 
t. h. kOORNWINDER, 
{\em tHE STRUCTURE RELATION FOR aSKEY-wILSON POLYNOMIALS}, 
J.~Comput.\ Appl.\ Math.\ 207 (2007), 214--226; {\tt ARxIV:MATH/0601303V3}. 
% 
\mybibitemKLS{382} 
p.  a. lESKY, 
{\em eNDLICHE UND UNENDLICHE sYSTEME VON KONTINUIERLICHEN KLASSISCHEN oRTHOGONALPOLYNOMEN}, 
z. aNGEW. mATH. mECH. 76 (1996), 181--184. 
% 
\mybibitemKLS{384} 
 p. a. lESKY, 
{\em eINORDNUNG DER pOLYNOME VON rOMANOVSKI-bESSEL IN DAS aSKEY-tABLEAU}, 
z.AaNGEW. mATH. mECH. 78 (1998), 646--648. 
% 
\mybibitemKLS{406} 
j. mEIXNER, 
{\em oRTHOGONALE pOLYNOMSYSTEME MIT EINER BESONDEREN gESTALT DER ERZEUGENDEN fUNKTION}, 
j. lONDON mATH. sOC. 9 (1934), 6--13. 
% 
\mybibitemKLS{416} 
a. f. nIKIFOROV, s. k. sUSLOV AND v. b. uVAROV, 
{\em cLASSICAL ORTHOGONAL POLYNOMIALS OF A DISCRETE VARIABLE}, 
sPRINGER-vERLAG, 1991. 
% 
\mybibitemKLS{449} 
m. rAHMAN, 
{\em sOME GENERATING FUNCTIONS FOR THE ASSOCIATED aSKEY-wILSON POLYNOMIALS}, 
j.AcOMPUT. aPPL. mATH. 68 (1996), 287--296. 
% 
\mybibitemKLS{463} 
v. rOMANOVSKI, 
{\em sUR QUELQUES CLASSES NOUVELLES DE polyn\^omes ORTHOGONAUX}, 
c. r. aCAD. sCI. pARIS 188 (1929), 1023--1025. 
% 
\mybibitemKLS{471} 
l. j. sLATER, 
{\em gENERALIZED HYPERGEOMETRIC FUNCTIONS}, cAMBRIDGE uNIVERSITY pRESS, 1966. 
% 
\mybibitemKLS{485} 
d. sTANTON, 
{\em a SHORT PROOF OF A GENERATING FUNCTION FOR jACOBI POLYNOMIALS}, 
pROC. aMER. mATH. sOC. 80 (1980), 398--400. 
% 
\mybibitemKLS{488} 
d. sTANTON, 
{\em oRTHOGONAL POLYNOMIALS AND cHEVALLEY GROUPS}, 
IN: {\em sPECIAL FUNCTIONS: GROUP THEORETICAL ASPECTS AND APPLICATIONS}, 
rEIDEL, 1984, PP. 87-128. 
% 
\mybibitemKLS{513} 
j. a. wILSON, 
{\em aSYMPTOTICS FOR THE ${}_4f_3$ POLYNOMIALS}, 
j. aPPROX. tHEORY 66 (1991), 58--71. 
% 
\end{thebibliography} 
% 
\makeatletter 
\renewcommand\@biblabel[1]{[K#1]} 
\makeatother 
% 
\renewcommand{\refname}{Other REFERENCES} 
\begin{thebibliography}{999} 
\label{sec_ref3} 
% 
% 
\bibitem{K16} 
r. aSKEY, 
{\em rAMANUJAN'S EXTENSIONS OF THE GAMMA AND BETA FUNCTIONS}, 
aMER. mATH. mONTHLY 87 (1980), 346--359. 
% 
\bibitem{K2} 
r. aSKEY AND j. fITCH, 
{\em iNTEGRAL REPRESENTATIONS FOR jACOBI POLYNOMIALS AND SOME APPLICATIONS}, 
j. mATH. aNAL. aPPL. 26 (1969), 411--437. 
% 
\bibitem{K24} 
m. n. aTAKISHIYEV AND v. a. gROZA, 
{\em tHE QUANTUM ALGEBRA $u_Q(SU_2)$ AND $Q$-kRAWTCHOUK FAMILIES OF 
POLYNOMIALS}, 
j. pHYS. a 37 (2004), 2625--2635. 
% 
\bibitem{K18} 
m. aTAKISHIYEVA AND n. aTAKISHIYEV, 
{\em oN DISCRETE $Q$-EXTENSIONS OF cHEBYSHEV POLYNOMIALS}, 
cOMMUN. mATH. aNAL. 14 (2013),  1--12. 
% 
\bibitem{K27} 
s. bELMEHDI, 
{\em gENERALIZED gEGENBAUER ORTHOGONAL POLYNOMIALS}, 
j. cOMPUT. aPPL. mATH. 133 (2001), 195--205. 
% 
\bibitem{K5} 
y. bEN cHEIKH AND m. gAIED, 
{\em cHARACTERIZATION OF THE dUNKL-CLASSICAL SYMMETRIC ORTHOGONAL POLYNOMIALS}, 
aPPL. mATH. cOMPUT. 187 (2007), 105--114. 
% 
\bibitem{K19} 
j. cIGLER, 
{\em a SIMPLE APPROACH TO $Q$-cHEBYSHEV POLYNOMIALS}, 
{\tt ARxIV:1201.4703V2 [MATH.co]}, 2012. 
% 
\bibitem{K23} 
pH. dELSARTE, 
{\em pROPERTIES AND APPLICATIONS OF THE RECURRENCE 
$f(I+1,K+1,N+1)=Q^{K+1}f(I,K+1,N)-Q^{K}f(I,K,N)$}, 
siam j. aPPL. mATH. 31 (1976), 262--270. 
% 
\bibitem{K26} 
c. f. dUNKL AND y. xU, 
{\em oRTHOGONAL POLYNOMIALS OF SEVERAL VARIABLES}, 
cAMBRIDGE uNIVERSITY pRESS, 2014, SECOND ED. 
% 
\bibitem{K3} 
e. fELDHEIM, 
{\em rELATIONS ENTRE LES POLYNOMES DE jACOBI, lAGUERRE ET hERMITE}, 
aCTA mATH. 75 (1942), 117--138. 
% 
\bibitem{K21} 
w. gAUTSCHI, 
{\em oN MEAN CONVERGENCE OF EXTENDED lAGRANGE INTERPOLATION}, 
j. cOMPUT. aPPL. mATH. 43 (1992), 19--35. 
% 
\bibitem{K25} 
v. x. gENEST, s. pOST, l. vINET, g.-f. yU AND a. zHEDANOV, 
{\em $Q$-rOTATIONS AND kRAWTCHOUK POLYNOMIALS}, 
ARxIV:1408.5292V2 [MATH-PH], 2014. 
% 
\bibitem{K28} 
v. x. gENEST, l. vINET AND a. zHEDANOV, 
{\em a "CONTINUOUS" LIMIT OF THE cOMPLEMENTARY bANNAI-iTO POLYNOMIALS: 
cHIHARA POLYNOMIALS}, 
sigma 10 (2014), 038, 18 PP.; ARxIV:1309.7235V3 [MATH.ca]. 
% 
\bibitem{K29} 
v. gORIN AND g. oLSHANSKI, 
{\em a QUANTIZATION OF THE HARMONIC ANALYSIS ON THE INFINITE-DIMENSIONAL 
UNITARY GROUP}, 
\href{http://arxiv.org/abs/1504.06832}{arXiv:1504.06832}v1 [MATH.rt], 2015. 
% 
\bibitem{K1} 
m. j. gOTTLIEB, 
{\em cONCERNING SOME POLYNOMIALS ORTHOGONAL ON A FINITE OR ENUMERABLE SET OF  POINTS}, 
aMER. j. mATH. 60 (1938), 453--458. 
% 
\bibitem{K14} 
w. gROENEVELT AND e. kOELINK, 
{\em tHE INDETERMINATE MOMENT PROBLEM FOR THE $Q$-mEIXNER POLYNOMIALS}, 
j. aPPROX. tHEORY 163 (2011), 836--863. 
% 
\bibitem{K11} 
k. jORDAAN AND f. To\'okos, 
{\em oRTHOGONALITY AND ASYMPTOTICS OF pSEUDO-jACOBI POLYNOMIALS FOR 
 NON-CLASSICAL PARAMETERS}, 
j. aPPROX. tHEORY 178 (2014), 1--12. 
% 
\bibitem{K7} 
t. h. kOORNWINDER, {\em aSKEY-wILSON POLYNOMIAL}, sCHOLARPEDIA 7 (2012), NO.A7, 
7761;\\ 
 \url{http://www.scholarpedia.org/article/Askey-Wilson_polynomial}. 
% 
\bibitem{K17} 
t. h. kOORNWINDER, 
{$Q$-sPECIAL FUNCTIONS, A TUTORIAL}, 
{\tt ARxIV:MATH/9403216V2 [MATH.ca]}, 2013. 
 % 
 \bibitem{K10} 
 n. n. lEONENKO AND n. \v{S}uvak, 
{\em sTATISTICAL INFERENCE FOR STUDENT DIFFUSION PROCESS}, 
sTOCH. aNAL. aPPL. 28 (2010), 972--1002. 
% 
\bibitem{K22} 
j. c. mASON, 
{\em cHEBYSHEV POLYNOMIALS OF THE SECOND, THIRD AND FOURTH KINDS IN 
APPROXIMATION, INDEFINITE INTEGRATION, AND INTEGRAL TRANSFORMS}, 
j. cOMPUT. aPPL. mATH. 49 (1993), 169--178. 
% 
\bibitem{K20} 
j. c. mASON AND d. hANDSCOMB, 
{\em cHEBYSHEV POLYNOMIALS}, 
cHAPMAN \& hALL / crc, 2002. 
% 
\bibitem{K9} 
j. mEIXNER, 
{\em uMFORMUNG GEWISSER rEIHEN, DEREN gLIEDER pRODUKTE HYPERGEOMETRISCHER 
fUNKTIONEN SIND}, 
dEUTSCHE mATH.  6 (1942),  341--349. 
% 
\bibitem{K4} 
n. nIELSEN, 
{\em rECHERCHES SUR LES polyn\^omes D' hERMITE}, 
kGL. dANSKE vIDENSK. sELSK. mATH.-fYS. mEDD. i.6, K\o BENHAVN, 1918. 
% 
\bibitem{K12} 
j. pEETRE, 
{\em cORRESPONDENCE PRINCIPLE FOR THE QUANTIZED ANNULUS, rOMANOVSKI  POLYNOMIALS, 
AND mORSE POTENTIAL}, 
j. fUNCT. aNAL.  117 (1993), 377--400. 
% 
\bibitem{K8} 
h. rOSENGREN, 
{\em mULTIVARIABLE ORTHOGONAL POLYNOMIALS AND COUPLING COEFFICIENTS FOR 
DISCRETE SERIES REPRESENTATIONS}, 
siam j. mATH. aNAL.  30  (1999),  233--272. 
% 
\bibitem{K13} 
e. j. rOUTH, 
{\em oN SOME PROPERTIES OF CERTAIN SOLUTIONS OF A DIFFERENTIAL EQUATION OF THE SECOND ORDER}, 
pROC. lONDON mATH. sOC. 16 (1885), 245--261. 
% 
\bibitem{K6} 
j. a. sHOHAT AND j. d. tAMARKIN, 
{\em tHE PROBLEM OF MOMENTS}, 
aMERICAN mATHEMATICAL sOCIETY, 1943. 
% 
\bibitem{K15} 
l. j. sLATER, 
{\em gENERAL TRANSFORMATIONS OF BILATERAL SERIES}, 
qUART. j. mATH., oXFORD sER. (2) 3 (1952), 73--80. 
% 
%UNTIL k29 
\end{thebibliography} 
\quad\\ 
\begin{footnotesize} 
\begin{quote} 
{t. h. kOORNWINDER, kORTEWEG-DE vRIES iNSTITUTE, uNIVERSITY OF Amsterdam,\\ 
P.O.\ bOX 94248, 1090 ge aMSTERDAM, tHE nETHERLANDS; 
 
\vspace{\smallskipamount} 
EMAIL: }{\tt t.h.kOORNWINDER@UVA.NL} 
\end{quote} 
\end{footnotesize} 
\end{document} 
